\documentclass[12pt]{article}
\usepackage{class/sbc-template}
\usepackage{graphicx,url}
\usepackage[utf8]{inputenc}
\usepackage[brazil]{babel}
\usepackage{multirow}
     
\sloppy

\title{Um estudo sobre as entradas de dados e a experiência do usuário no desenvolvimento de sistemas computacionais para a Agricultura 4.0}

\author{Diogo C. T. Batista\inst{1}, Cléber G. Corrêa\inst{1}, Letícia M. Peres\inst{2}, Roberto Pereira\inst{2}}

\address{Universidade Tecnológica Federal do Paraná (UTFPR)\\
  Cornélio Procópio -- Paraná -- Brasil
	\nextinstitute
	Universidade Federal do Paraná (UFPR)\\
	Curitiba -- Paraná -- Brasil
	\email{diogo@diogocezar.com,clebergimenez@utfpr.edu.br,\{lmperes,rpereira\}@inf.ufpr.br}
	}

\begin{document} 

\maketitle
     
\begin{resumo} 
A Agricultura 4.0 explora a utilização de tecnologias computacionais de ponta, envolvendo a Agricultura e Pecuária de Precisão, bem como a Agricultura Digital, que empregam a automação, a robótica agrícola, \textit{Big Data} e a Internet das Coisas, entre outras. A utilização dessas tecnologias busca uma produção agrícola eficiente e sustentável, possibilitando, por exemplo, a economia de água na irrigação ou de insumos na adubagem dos solos. Entretanto, o acesso aos recursos necessários para a exploração dessas tecnologias não é a realidade de grande parte do setor agrícola. Além disso, a resistência na adoção de novas tecnologias é um problema ainda em aberto. Este trabalho busca explorar soluções acessíveis a grande parte da cadeia agrícola, propondo um estudo direcionado ao aumento da eficácia na obtenção de dados a serem usados por sistemas computacionais. Mesmo em sistemas agrícolas de alta tecnologia, com aceitação dos usuários, as entradas dos usuários em campo por meio de \textit{smartphones}, e os tipos dessas entradas (imagens, voz, vídeo, modelos tridimensionais), ainda são negligenciadas no desenvolvimento desses sistemas. A preferência é por entradas provenientes de sensores (por exemplo, veículos aéreos não tripulados, satélites) e pela visualização das informações (saídas do sistema) para tomada de decisão. Dessa forma, deve ser realizado um estudo relacionado as informações de entradas do sistemas, especialmente por parte dos usuários de \textit{smartphones}. Uma solução pode ser por meio de aprimoramentos nas interfaces dos sistemas, não somente interfaces gráficas, melhorando a experiência do usuário, e principalmente, incluindo metadados empíricos, que aproximam o usuário do processamento computacional e consideram sua experiência prévia nos resultados a serem inferidos, para o apoio a tomada de decisão.
\end{resumo}

\section{Introdução}
\label{sec:introducao}

Serão grandes os desafios relacionados à utilização de tecnologias digitais como ferramentas de apoio ao trabalho humano de forma eficiente, eficaz e justa \cite{Rose:2019}. Para isso, a humanidade apoia-se nas descobertas e comprovações de milhares de anos da comunidade científica, que proporcionam resultados relevantes, alcançados por meio da exploração das técnicas e métodos descobertos e aprimorados ao longo dos anos.

A indústria é um exemplo de como essa evolução aconteceu. Seu marco se dá com a definição da Era Artesanal, na qual, os bens e serviços eram criados por um único indivíduo: o artesão. Este era responsável por deter todo o conhecimento dos métodos e processos para a fabricação de um produto ou para a prestação de um serviço. Com a criação e o aperfeiçoamento das máquinas a vapor, estes processos puderam ser minimamente automatizados, e isso deu origem a Primeira Revolução Industrial. A Segunda Revolução Industrial, acontece com a utilização da eletricidade e principalmente dos processos que possibilitaram a implantação da produção em massa idealizadas por Henry Ford. Já a Terceira Revolução Industrial se inicia após a Segunda Guerra Mundial, com o descobrimento e a utilização da robótica e do uso de computadores para a automação das indústrias \cite{Souza:2017}.

Mais recente, muito tem se falado sobre a Quarta Revolução Industrial, que surge em 2011 na Alemanha, com a proposta de oferecer para a indústria o que há de mais moderno em automação e sistemas inteligentes, possibilitando uma série de melhorias como por exemplo: a redução dos custos, a economia de energia e o aumento da segurança. Essas e outras melhorias têm sido exploradas por meio da utilização de ferramentas e tecnologias como: \textit{Big Data}, \textit{Analytics}, Serviços de Nuvem, Impressões 3D, Segurança Cibernética, Robôs Autônomos, Internet das Coisas, Sensores sem Fio, Realidade Aumentada, Simulação, Integração Horizontal e Integração Vertical \cite{Souza:2017}.

Além da indústria, outro setor que evoluiu muito ao longo dos anos foi a agricultura. A agricultura foi fundamental para alavancar o desenvolvimento das civilizações ao longo da história, possibilitando que comunidades nômades se estabilizarem em determinadas regiões, explorando técnicas agrícolas para a produção de seu sustento. Entretanto, o cenário atual é bem diferente dos enfrentados pelas primeiras civilizações, o crescimento populacional, e a tendência das pessoas a se tornarem cada vez mais exigentes com o que é consumido, gera um grande desafio para o setor agrícola, que precisa se posicionar eficientemente para atender estas demandas. Dessa forma, novas técnicas têm sido exploradas, e estas, compõem o termo Agricultura 4.0. Estas técnicas utilizam os mesmos métodos e processos inovadores já explorados pela Indústria 4.0, incluindo: a automação, a robótica agrícola, \textit{Big Data} e Internet Das Coisas, em um contexto de Agricultura e Pecuária de Precisão e Agricultura Digital \cite{Ribeiro:2018}.

A utilização de componentes tecnológicos não é uma novidade na Agricultura, a exploração, por exemplo, de tecnologias como o \textit{Global Positioning System} (GPS), estão presentes a algum tempo na linha de produção agrícola, conhecida como Agricultura de Precisão. A Agricultura 4.0 expande as possibilidades de explorações tecnológicas, com a utilização por exemplo de: \textbf{Sensores}; que possibilitam a coleta automática de dados sobre o solo e clima; \textbf{Veículos Aéreos Não Tripulados (drones) ou Satélites}, que podem apresentar recursos de imagem cada vez mais avançados, auxiliando no aumento de produtividades e ajudando a reduzir danos nas lavouras, visto que possibilitam o monitoramento em tempo real; \textbf{smartphones}, que proporcionam uma interface de entrada e saída de dados rápida, acessível e já conhecida por seu usuário \cite{Shepherd:2018}.

A Agricultura 4.0, com a utilização de sistemas de apoio à decisões, pode levar à produções mais eficientes e ao consumo mais inteligente e sustentável. Entretanto, essa transição não é uma tarefa simples. Segundo \cite{Rose:2019}, a utilização destas ferramentas tecnológicas devem mudar a forma como os agricultores interagem com suas plantações, com isso, cada vez menos os trabalhadores precisão colocar a ``mão na massa'', mudando a forma como sempre interagiam com seu ofício. E isso pode dificultar a adesão de novas tecnologias. A utilização das técnicas relacionadas a agricultura de precisão demonstraram certa resistência por parte dos agricultores \cite{Rose:2019}. Entretanto, o uso em larga escala de Inteligência Artificial (IA), robótica e outras inovações emergentes, tem o claro potencial de causar consequências sociais não intencionais, imprevistas e indesejadas. Por este motivo, é importante focar em uma parte fundamental para o sucesso da aplicação de novas tecnologias: o usuário.

Grande parte das tecnologias adotadas pela Agricultura 4.0 estão inseridas (pelo menos em algum momento) em um sistema computacional. Sabe-se que um sistema computacional é basicamente composto pela \textit{entrada de dados}, o seu \textit{processamento} e a disponibilização dos seus \textit{resultados}. As duas últimas etapas dependem da qualidade da fonte de dados ou seja da entrada das informações \cite{Torres:2013}.

No âmbito agrícola, por mais automatizados que possam estar os processos de obtenções dos dados, com sensores automáticos ou análise de imagens, por exemplo, há dois principais pontos que podem aumentar ainda mais a resistência à adoção das tecnologias: \textit{Indisponibilidade de Equipamentos} ou \textit{Conhecimento Empírico}. No cenário atual, não são todos os agricultores que podem destinar parte do seu faturamento para a aquisição de equipamentos modernos que facilitariam a coleção de dados. Além disso, existe uma variável de grande importância que deve ser inserida na equação da Agricultura 4.0, a experiência do agricultor com a sua propriedade. Segundo \cite{Rose:2019} aproximar o usuário final do processo tecnológico, pode ser uma estratégia eficiente para a adoção de novas tecnologias.

Dessa forma, o presente projeto pode contribuir com determinados \textit{Objetivos de Desenvolvimento Sustentável}, descritos pela Nações Unidas \cite{UN:2020}. Neste documento, descreve-se 17 objetivos que buscam concretizar os direitos humanos e promover a igualdade. Para isso, tais objetivos e metas são descritos com o intuito de estimular ações para os próximos 15 anos, em áreas de importância crucial para a humanidade e para o planeta. Dentre os objetivos, dois deles tem relação direta com este trabalho:

\begin{itemize}
	\item \textit{Objetivo 2: Fome zero e agricultura sustentável} - Aumentar a eficiência da produção agrícola é consequência direta da proposta deste trabalho. A criação de interfaces inclusivas e que consideram a experiência profissional do usuário podem proporcionar dados de entradas mais refinados, que consequentemente, podem melhorar os resultados de apoio a decisão.
	\item \textit{Objetivo 12: Consumo e produções sustentáveis} - Neste quesito, o trabalho contribui com a proposta de uma solução eficiente, acessível (utilizando um \textit{smartphone} como equipamento, por exemplo) e sem a necessidade da aquisição de novos equipamentos para integrações tecnológicas.
\end{itemize}

\section{Objetivo}
\label{sec:objetivo}

O objetivo é analisar as informações de entradas de usuários de dispositivos móveis que atuam em zonas agrícolas, especialmente \textit{smartphones}, para o desenvolvimento de sistemas computacionais no âmbito da Agricultura 4.0, incluindo sistemas de suporte a decisão agrícolas. As entradas não devem ser somente texto, com anotações e preenchimento de formulário de atributos, mas também: imagens, discursos (reconhecimento de voz), modelos tridimensionais, vídeos e etc; bem como combinações de entradas, como imagens com anotações textuais, ou sobreposições de destaques em áreas da imagem, permitindo maior interações com as imagens e demais entradas de dados. Os vídeos podem ser utilizados para registrar uma determinada sequência de eventos, descrevendo um fenômeno; modelos tridimensionais podem proporcionar detalhes de algum elemento da cultura (por exemplo, uma planta de soja) e podem ser construídos com imagens de diferentes ângulos de um objeto alvo. Outro ponto é a utilização de dispositivos, como os \textit{smartphones}, que se tornaram populares e possuem recursos para suportar essas entradas, como câmera e microfone.

Como pode ser visto em \cite{Massruha:2017}, e nos trabalhos relacionados (Seção \ref{sec:trabalhos_relacionados}), o foco de sistemas computacionais na Agricultura 4.0 está nos resultados obtidos pela análise das informações obtidas por meio das saídas desses sistemas. No cenário proposto pela Agricultura 4.0 grande parte dos trabalhos entendem que as informações que alimentam os sistemas, podem ser obtidas de forma automatizada, através de sensores ou pela análise de imagens, por exemplo, entretanto nem sempre essa é uma realidade aplicável a todos, pelo alto custo e difícil aderência. Outros sistemas, utilizam ainda, apenas entrada manual de informações por parte dos seus usuários, normalmente através de formulários com campos pré definidos (comumente dados no formato texto).

Complementando o objetivo do presente trabalho, a análise deve permitir a criação de métodos de entrada de dados que levem em consideração não somente a transcrição exata do estado atual observado pelo coletor das informações, mas também os seus \textit{insigts} com base em sua experiência e observação. A análise deve considerar a necessidade dessas entradas, como forma de usar a experiência dos usuários em campo pelo sistema; o desenvolvimento das interfaces de \textit{smartphones} para essas entradas (gráficas e sintetizadores de voz); a diversidade de usuários (agricultores, técnicos agrícolas, engenheiros agrônomos); a relação dessas entradas com outras informações (imagens, temperatura, umidade, biomassa etc), provenientes de sensores, como dispositivos posicionados na plantação, imagens de satélites, veículos aéreos não tripulados, etc, podendo complementar o processamento de informações.

\section{Trabalhos Relacionados}
\label{sec:trabalhos_relacionados}

Na exploração dos trabalhos relacionados, fica evidente que a maioria dos trabalhos que abordam às inovações da Agricultura 4.0 focam nas saídas de dados (visualização de informações), e entradas provenientes de sensores, permitindo um processamento para o apoio a tomada de decisão na produção agrícola. São poucas as explorações com relação ao refinamento das entradas de dados. 

O trabalho de \cite{Zhai:2020}, apresenta como desafios para os sistemas de apoio a decisão: a simplificação de interfaces gráficas dos usuários e utilização de conhecimento dos especialistas do domínio do sistema. Além disso, são abordadas alternativas às entradas de dados comuns, como por exemplo: a utilização do reconhecimento de voz ou a utilização gestos. Outro ponto importante é relacionado ao processo de extração das informações da fonte de dados. Modelos tridimensionais, análise de imagens ou vídeos, e imagens interativas podem ser pontos ainda a serem explorados no contexto em que este trabalho se aplica. Este tema relaciona-se com o trabalho proposto validando a necessidade do aprimoramento das interfaces utilizadas para a extração dos dados de forma eficiente e agradável.

\cite{Gutierrez:2019}, demonstra através da comparação de trabalhos, que a utilização de diversas fontes de informações de entradas é um ponto desejável para a automatização e autonomia na Agricultura 4.0. As principais comparações estão relacionadas às formas de visualização de dados já processados. As entradas dos dados são tratadas de forma superficial. Este trabalho reafirma a preocupação do presente trabalho em inserir diversas dimensões em uma mesma fonte de dados, e que o processo de obtenção de dados refinados pode aprimorar os resultados obtidos.

Nos trabalhos de \cite{Walling:2020} e \cite{Lundstrom:2018} demonstra-se a importância da participação do usuário no processo de obtenção dos dados. Profissionais com a experiência de seu domínio, são capazes de enriquecer toda a cadeira de inferência nos sistemas de apoio a decisão. Com este trabalho, fica claro que a participação multidisciplinar em toda a cadeia dos sistemas de apoio a decisão podem fazer uma diferença expressiva nos resultados.

O sistema AgroDSS, explorado por \cite{Rupnik:2019} é um exemplo de sistema elegível às melhorias propostas por este trabalho. Nesta abordagem, a maioria dos dados coletados é baseado apenas em informações inseridas por meio de formulários de atributos textuais.

O sistema Adama Alvo \footnote{Disponível em: https://www.adama.com/brasil/pt/adama-inovacao/adama-alvo} é um aplicativo para auxiliar na identificação de doenças, pragas e ervas daninhas em diferentes culturas. Um dos seus principais objetivos é fornecer uma base de dados offline para o usuário, de forma que seja possível analisar e comparar em tempo real uma praga encontrada em uma folha com uma foto da mesma. Além disso, o aplicativo permite o envio de imagens por parte do usuário para serem analisadas pelos especialistas da empresa Adama, que posteriormente podem catalogá-la e disponibilizá-la no aplicativo, auxiliando outros agricultores.

Na Figura \ref{fig:adamalvo} pode-se observar a interface do sistema Adama Alvo. Pela interface do aplicativo, é possível navegar entre categorias. Na demonstração, uma coleção de lagartas é exibida. Também nota-se como é possível obter os detalhes de um inseto específico (Percevejo verde) e como o Aplicativo oferece o apoio ao usuário mostrando a previsão do tempo.

\begin{figure}[!htb]
	\centering
  \includegraphics[scale=0.5]{images/AdamaAlvo.png}
  \caption{Interfaces gráficas do sistema Adama Alvo em diferentes situações de uso}
  \label{fig:adamalvo}
\end{figure}

\section{Análise Crítica e Relevância}
\label{sec:analise_critica_relevancia}

No cenário agrícola, são vários os papéis dos profissionais que compões toda a cadeia de produção. Entretanto, a obtenção dos dados utilizados nos sistemas computacionais de apoio à agricultura, quando inseridas de forma manual, normalmente, levam em consideração apenas uma única fonte de informações, que quase sempre é o próprio agricultor. Apesar dos possíveis conhecimentos coletados, a informação poderia ser enriquecida, se neste processo, houvesse por exemplo a visão de outros usuários. Por exemplo, para uma nova praga identificada em uma folha de soja, o \textbf{agricultor} poderia inserir as informações relacionadas as suas experiências, como por exemplo se já houve alguma praga com indícios semelhantes aos encontrados, a quanto tempo, por quanto tempo, e como foi resolvido. Enquanto que o \textbf{engenheiro agrônomo} poderia inserir para esta mesma entrada informações extraídas de sua experiência em outras fazendas, por exemplo. Dessa forma consegue-se extrair diversas \textbf{dimensões} a partir de uma única entrada de dados \cite{Walling:2020}.

Segundo uma pesquisa divulgada pela FGV-SP \footnote{Disponível em: https://eaesp.fgv.br/ensinoeconhecimento/centros/cia/pesquisa} o Brasil tem hoje dois dispositivos digitais por habitante, incluindo \textit{smartphones}, computadores, \textit{notebooks} e \textit{tablets}. Em 2019, o país teria 420 milhões de aparelhos digitais ativos. Dessa forma, fica bastante claro que estes dispositivos já fazem parte do dia a dia de muitas pessoas, inclusive dos que fazem parte de toda a cadeia agrícola.

A utilização de \textit{smartphones} e computadores, propõe uma solução acessível para um dos pontos em \cite{Rose:2019}, que argumenta, que é necessário ampliar as noções de ``inclusão'' em inovação responsável para atender às diversas formas de como os agricultores devem interagir com as fazendas inteligentes. Estes dispositivos já fazem parte da rotina de trabalho do público alvo, não sendo necessários outros investimentos ou aquisições, e ainda assim, possibilitando a inclusão deste público em novas ferramentas criadas para a Agricultura 4.0.

\subsection{Cenário Possível}
\label{sec:cenario_possivel}

Para ilustrar as possibilidades a serem exploradas, descreve-se um cenário hipotético imaginado no qual:

\begin{enumerate}
	\item Um usuário, motivado pelo desconhecimento de uma praga, captura uma imagem de um inseto em uma planta de soja usando seu \textit{smartphone};
	\item A imagem pode receber informações adicionais do usuário no próprio \textit{smartphone}, tais como: anotações da quantidade, nome do inseto, enfatizando o tamanho do inseto e o estádio fenológico da planta;
	\item Ainda na imagem, pode ser possível a inclusão de anotações ou destaques em pontos que pudessem ser circulados, pelo usuário, demonstrando pontos de atenção que pudessem ser analisados;
	\item Duas imagens poderiam servir de fonte de dados: a original e a com as anotações do usuário;
	\item Por conta de uma possível dificuldade de conexão (muito comum no cenário rural) o envio das informações pode não acontecer imediatamente. Mas a operação pode ser marcada como sucesso, deixando essa atividade uma fila de execução em \textit{background};
	\item O próprio sistema registra as coordenadas de geolocalização bem como e o momento (horário/dia) desta captura ou oferece a possibilidade de inserção manual destas informações (prevendo a instabilidade da conexão);
	\item Ao receber as informações, um sistema computacional em nuvem, pode relacionar as imagens recebidas com uma ou mais imagens aéreas, capturadas por veículos aéreos não tripulados da mesma região onde está a planta ou regiões adjacentes em um ou mais momentos anteriores, buscando por insetos. Além disso, um outro ponto importante é que a base de conhecimento pode ser compartilhada, enriquecendo as informações a serem inferidas.
\end{enumerate}

\section{Métodos e Materiais}
\label{sec:metodos_materiais}
		
Os métodos para a realização da pesquisa incluem:

\begin{itemize}
	\item Projeto e desenvolvimento de protótipos de sistemas computacionais (aplicativos) para validação;
	\item Planejamento, organização e execução de experimentos envolvendo usuários, normalmente com a utilização de questionários para coletar opiniões referentes as percepções desses usuários na interação humano-computador. Informações como número de cliques, tempo de uso, também podem ser usadas. No âmbito de experimentos com usuários, será realizada a submissão de projeto para avaliação de Comitê de Ética em Pesquisa com Seres Humanos da UTFPR\footnote{Disponível em: http://www.utfpr.edu.br/comissoes/permanentes/comite-de-etica-em-pesquisa} (CEP-UTFPR), utilizando a Plataforma Brasil, do Ministério da Saúde\footnote{Disponível em: http://plataformabrasil.saude.gov.br/login.jsf};
	\item Revisão da literatura, podendo utilizar métodos formais de pesquisa, como a revisão sistemática, que fornece um protocolo para o levantamento bibliográfico \cite{Kitchenham:2004};
	\item Análise de resultados utilizando testes estatísticos para verificar a significância das diferenças entre os grupos amostrais comparados (Friedman, ANOVA – Análise da Variância, t-test, Wilcoxon, Mann-Whitnney etc); e estatística descritiva (principalmente gráficos). Os testes estatísticos dependem das características dos dados (normalidade da distribuição, homogeneidade das variâncias, independência e aleatoriedade na coleta etc) e oferecem intervalos de confiança para a análise.
\end{itemize}

Os materiais consistem em \textit{smartphones}, dos próprios usuários; bem como serviços de nuvem para processamento das informações adicionadas pelos usuários. Uma avaliação da aquisição de determinados sensores para medição de informações como temperatura e umidade do solo, será realizada. Adicionalmente, informações de terceiros podem ser empregadas, tais como imagens de satélites e de sítios eletrônicos sobre o clima. 		

A cultura foco provavelmente será a de soja, devido a parceria da UTFPR com a Empresa Brasileira de Pesquisa Agropecuária (Embrapa), situada em Londrina, que trabalha principalmente com a produção da soja. A Embrapa é uma empresa pública de pesquisa vinculada ao Ministério da Agricultura, Pecuária e Abastecimento do Brasil. O seu principal objetivo é a viabilização de soluções provenientes de pesquisas, além do desenvolvimento para a sustentabilidade da agricultura em benefício da sociedade Brasileira. 

A parceria atualmente visa o desenvolvimento de dois sistemas:

\begin{itemize}
	\item Sistema para identificação de pragas, doenças e plantas daninhas da cultura de soja, com suporte \textit{web} e dispositivos móveis. A utilização de imagens capturadas pelos usuários é um dos requisitos do sistema;

	\item Sistema para acompanhamento do crescimento da planta de soja, apresentando os estádios fenológicos, com suporte \textit{web} e dispositivos móveis. A utilização de modelos tridimensionais e vídeos para saída de informações ou visualização por parte dos usuários está prevista. Exemplo de alguns estádios fenológicos da soja podem ser observados na Figura \ref{fig:estadios_fenelogicos}.
\end{itemize}

\begin{figure}
  \includegraphics[scale=0.5]{images/EstadiosFenologicos.png}
  \caption{Alguns dos estádios fenológicos da planta da soja (Embrapa)}
  \label{fig:estadios_fenelogicos}
\end{figure} 

Esses sistemas ou módulos desses sistemas poderão ser empregados em estudos de caso, servindo como protótipos para validação, se necessários. 

\section{Plano de Trabalho e Cronograma}
\label{sec:plano_trabalho_cronograma}

O plano de trabalho é composto pelas seguintes atividades principais, dispostas no cronograma apresentado na Tabela \ref{tab:cronograma}, separadas por quadrimestres de cada ano:

\begin{enumerate}
	\item Obtenção de créditos, cursando disciplinas;
	\item Estudo de métodos para mapeamento ou revisão sistemática, permitindo o levantamento de trabalhos na literatura;
	\item Estudo do estado da arte, considerando sistemas computacionais para dispositivos móveis voltados para a agricultura 4.0 e seus usuários;
	\item Levantamento de requisitos, especialmente funcionalidades relacionadas as entradas de dados dos usuários;
	\item Exame de proficiência de língua estrangeira;
	\item Exame de qualificação;
	\item Especificação de um sistema computacional para dispositivos móveis, indicando a arquitetura do sistema e principais desafios no desenvolvimento;
	\item Implementação de protótipo para validação;
	\item Elaboração de documentação para Comitê de Ética em Pesquisa com Seres Humanos da UTFPR;
	\item Testes envolvendo usuários;
	\item Redação de artigos científicos a serem submetidos aos principais eventos e periódicos da área de interesse;
	\item Redação da tese;
	\item Defesa.
\end{enumerate}

\begin{table}[htbp]
	\centering
	\begin{tabular}{|c|c|c|c|c|c|c|c|c|c|c|c|c|}
	\hline
	\multirow{2}{*}{AT} & \multicolumn{3}{c|}{\textbf{Ano 1}} & \multicolumn{3}{c|}{\textbf{Ano 2}} & \multicolumn{3}{c|}{\textbf{Ano 3}} & \multicolumn{3}{c|}{\textbf{Ano 4}} \\ \cline{2-13} 
											& Q1         & Q2         & Q3        & Q1         & Q2         & Q3        & Q1         & Q2         & Q3        & Q1         & Q2         & Q3        \\ \hline
	1                   & •          & •          & •         &            &            &           &            &            &           &            &            &           \\ \hline
	2                   &            &            &           & •          &            &           &            &            &           &            &            &           \\ \hline
	3                   &            &            &           & •          & •          & •         & •          &            &           &            &            &           \\ \hline
	4                   &            &            &           &            & •          &           &            &            &           &            &            &           \\ \hline
	5                   &            &            &           &            & •          &           &            &            &           &            &            &           \\ \hline
	6                   &            &            &           &            & •          &           &            &            &           &            &            &           \\ \hline
	7                   &            &            &           &            & •          &           &            &            &           &            &            &           \\ \hline
	8                   &            &            &           &            &            & •         & •          & •          &           &            &            &           \\ \hline
	9                   &            &            &           &            &            &           &            &            & •         &            &            &           \\ \hline
	10                  &            &            &           &            &            &           &            &            &           & •          &            &           \\ \hline
	11                  &            &            &           &            &            & •         & •          & •          & •         & •          & •          & •         \\ \hline
	12                  &            &            &           &            &            &           &            &            & •         & •          & •          & •         \\ \hline
	13                  &            &            &           &            &            &           &            &            &           &            &            & •         \\ \hline
	\end{tabular}
	\caption{Cronograma de Execução das Atividades}
	\label{tab:cronograma}
\end{table}

Com o objetivo de manter a constante comunicação, e principalmente o acompanhamento e direcionamento das atividades, reuniões semanais devem ser realizadas. Estas reuniões podem ser realizadas de forma presencial (quando possível) ou por meio de vídeo chamadas.

\section{Resultados Esperados}
\label{sec:resultados_esperados}

Pretende-se com a análise das entradas de informações dos usuários, verificar a importância e o impacto dessas informações nos sistemas computacionais para a Agricultura 4.0. Conforme a literatura, há um maior interesse nas saídas de informações para os usuários em sistemas computacionais para a agricultura, especialmente nos sistemas de tomada de decisão. Outros tipos de informações, como imagens, voz, modelos tridimensionais e vídeos, inseridos pelos usuários em campo por meio de \textit{smartphones}, podem ser explorados, devido a popularização e recursos desses dispositivos.

Espera-se, por fim, a submissão de artigos para conferências e periódicos científicos da área de Ciência da Computação para publicação, como o \textit{ACM Computing Surveys}, \textit{Computer Graphics and Applications}, \textit{Journal on Interactive Systems}, \textit{Computers and Electronics in Agriculture}, \textit{Symposium on Applied Computing}, Simpósio Brasileiro de Engenharia de Software. 

\bibliographystyle{sbc}
\bibliography{pre-project}

\end{document}