\documentclass[12pt]{article}
\usepackage{class/sbc-template}
\usepackage{graphicx,url}
\usepackage[utf8]{inputenc}
\usepackage[brazil]{babel}
     
\sloppy

\title{Um estudo sobre as entradas de dados e a experiência do usuário na construção de sistemas para a Agricultura 4.0}

\author{Diogo Cezar Teixeira Batista\inst{1}, Cleber Gimenez Corrêa\inst{1}}

\address{Departamento de Computação -- Universidade Tecnológica Federal do Paraná
  (UTFPR)\\
  Cornélio Procópio -- Paraná -- Brazil
  \email{diogoctb@gmail.com, clebergimenez@utfpr.edu.br}
}

\begin{document} 

\maketitle

\begin{abstract}
  abstract
\end{abstract}
     
\begin{resumo} 
  resumo
\end{resumo}

\section{Introdução}

Serão grandes os desafios relacionados à utilização de tecnologias digitais como ferramentas de apoio ao trabalho humano de forma eficiente, eficaz e justa \cite{agriculture-40-rose}. Para isso, a humanidade apoia-se nas descobertas e comprovações de milhares de anos da comunidade científica, que proporcionam resultados inimagináveis, alcançados através da exploração das técnicas e métodos definidos e aprimorados ao longos dos anos.

A indústria é um exemplo de como essa evolução aconteceu. Seu marco acontece com a definição da Era Artesanal, na qual, os bens e serviços eram criados por um único indivíduo: o artesão. Este era responsável por deter todo o conhecimento dos métodos e processos para a fabricação do produto ou a prestação de um serviço. Com a criação e o aperfeiçoamento das máquinas a vapor, estes processos puderam ser minimamente automatizados, e isso deu origem a Primeira Revolução Industrial. A Segunda Revolução Industrial, acontece com a utilização da eletricidade e principalmente dos processos que possibilitaram a implantação produção em massa idealizadas por Henry Ford. Já a Terceira Revolução Industrial se inicia após a Segunda Guerra Mundial, com o descobrimento e a utilização da robótica e do uso de computadores para a automação das indústrias. \cite{industria-40}

Mais recente, muito tem se falado sobre a Quarta Revolução Industrial, que surge em 2011 na Alemanha, com a proposta de oferecer para a indústria o que há de mais moderno em automação e sistemas inteligentes, possibilitando uma série de melhorias como por exemplo: a redução dos custos, a economia de energia e o aumento da segurança. Essas e outras melhorias têm sido exploradas através da utilização de ferramentas e tecnologias como: Big Data, Analytcs, Serviços de Núvem, Impressões 3D, Segurança Cibernética, Robôs Autônomos, Internet das Coisas, Sensores sem Fio, Realidade Aumentada, Simulação, Integração Horizontal e Integração Vertical. \cite{industria-40}

Além da Indústria, outro setor que evoluiu muito ao longo dos anos foi a Agricultura. Esta, foi fundamental para alavancar o desenvolvimento das civilizações ao longo da história, possibilitando que comunidades nômades se estabilizassem em determinadas regiões, explorando técnicas agrícolas para a produção de seu sustento. Entretanto, o cenário atual é bem diferente dos enfrentados pelas primeiras civilizações, o crescimento populacional, e a tendência das pessoas a se tornarem cada vez mais exigentes com o que é consumido, gera um grande desafio para o setor agrícola, que deve se posicionar para atender tais demandas. Desta forma, novas técnicas vêm sido exploradas, estas, compõem o termo Agricultura 4.0, ou Agricultura Digital. Estas utilizam os mesmos métodos e processos inovadores já explorados pela Indústria 4.0, incluindo: Agricultura e Pecuária de Precisão, a automação e a robótica agrícola, além de outras técnicas como Big Data e Internet Das Coisas. \cite{agricultura-40}

A utilização de componentes tecnológicos não é uma novidade na Agricultura, a exploração, por exemplo, de tecnologias como o GPS, estão presentes já a algum tempo na linha de produção agrícola, conhecida como Agricultura de Precisão. Já a Agricultura 4.0 expande as possibilidades de explorações tecnológicas, com a utilização por exemplo de: \textit{Sensores} que possibilitam a coleta autmática de dados sobre o solo, \textit{Drones ou Satélites} que podem apresentar recursos de imagem cada vez mais avançados, auxiliando no aumento de produtividades e ajudando a reduzir danos nas lavouras, uma vez que possibilitam o monitoramento em tempo real, \textit{SmartPhones} que proporcionam uma interface de entrada de dados rápida, acessível e já conhecida por seu usuário. \cite{digital-agriculture}

Não há dúvidas de que a tecnologia pode colocar a agricultura em seu próximo nível, A Agricultura 4.0 de aliada à produções mais variadas e a utilização de sistemas de apoio à decisões podem levar à produções mais eficientes e ao consumo mais inteligente e sustentável. Entretanto, essa transição não é uma tarefa simples. Segundo cite{agriculture-40-rose}, a utilização destas ferramentas tecnológicas devem mudar a forma como os agricultores interagem com suas terras, com isso, cada vez menos os trabalhadores precisão colocar a mão na massa, mudando a forma como sempre interagiam com seu ofício. E isso pode dificultar a adesão de novas tecnologias. A utilização das técnicas relacionadas a agricultura de precisão demonstraram certa resistência por parte dos agricultores. Mas o uso em larga escala de IA, robótica e outras inovações emergentes, tem o claro potencial de causar consequências sociais não intensionais, imprevistas e indesejadas. Por este motivo, é importante focar em uma peça fundamental para o sucesso da aplicação de novas tecnologias: o usuário.

Grande parte das tecnologias adotadas pela Agricultura 4.0 estão inseridas (pelo menos em algum momento) em um sistema computacional. Sabe-se que um sistema computacional é basicamente composto pela \textit{entrada de dados}, o seu \textit{processamento} e a disponibilização dos seus \textit{resultados}. As duas últimas etapas, por mais refinadas que estejam, dependem da qualidade da fonte de dados ou seja da entrada das informações.

No âmbito agrícola, por mais automatizados que possam estar os processos de obtenções dos dados, com sensores automáticos ou análise de imagens, por exemplo, temos dois principais pontos que podem aumentar ainda mais a resistência à adoção das tecnologias: \textit{Indisponibilidade de Equipamentos} ou \textit{Conhecimento Empírico}. No cenário atual, não são todos os agricultores que podem destinar parte do seu faturamento para a aquisição de equipamentos modernos que facilitariam a coleção de dados. Além disso, existe uma variável de grande importância que deve ser inserida na equação da Agricultura 4.0, a experiência do agricultor com a sua propriedade.

Por este motivo, a idéia deste trabalho é proporcionar um estudo direcionado aos métodos de obtenção dos dados, considerando técnidas de interface e usabilidade, para propor e experimentar diretrizes que apoiem o usuário (agricultor) através de interfaces com dispositivos computacionais. Possibilitando não somente a inserção dos dados observados, mas também a consideração dos \textit{insigts} obtidos por meio de sua experiência, possibilitando o refinamento dos dados inseridos nos sistemas computacionais.

\section{Abaixo estão alguns refinamentos dos tópicos compartilhados para serem ajustados ao longo do texto}

\subsection{Cultura Foco e Embrapa}

A Empresa Brasileira de Pesquisa Agropecuária (Embrapa) é uma empresa pública de pesquisa vinculada ao Ministério da Agricultura, Pecuária e Abastecimento do Brasil. O seu principal objetivo é a viabilização de soluções proveniente de pesquisas, além do desenvolvimento para a sustentabilidade da agricultura em benefício da sociedade Brasileira. Uma parceria firmada entre a Embrapa Londrina e a Universidade Tecnológica Federal do Paraná (UTFPR) possibilita a interação direta com pesquisadores e especialistas facilitando a validação e o acompanhamento dos experimentos para este trabalho.

Afim de restringir o domínio de atuação deste trabalho, os experimentos serão restritos ao sub-domínio da cadeia de produção da soja.

\subsection{Análise sobre as Entradas de Dados}

Como pode ser visto em \cite{rumo-agricultura-digital}, muitos dos trabalhos compilados, focam nos resultados obtidos pela análise das informações obtidas através das saídas do sistemas computacionais. No cenário proposto pela Agricultura 4.0 grande parte dos trabalhos entendem que as informações que alimentam os sistemas, podem ser obtidas de forma automatizada, através de sensores ou pela análise de imagens, por exemplo, entretanto nem sempre essa é uma realidade aplicável a todos, pelo alto custo e difícil aderência.

Outros sistemas, utilizam ainda, a entrada manual de informações por parte dos seus usuários, normalmente através de formulários com campos pré definidos (normalmente dados no formato texto).

Uma das propostas deste trabalho é analisar e propor métodos de entrada de dados que levem em consideração não somente a transcrição exata do estado atual observado pelo coletor das informações, mas também os seus insigts com base em sua experiência e observação. Além disso, a forma de inserção dos dados (quando de forma manual) será estudada, para possibilitar a aplicação destes insights.

\subsection{Adama Alvo}

Adama Alvo \footnote{Disponível em: https://www.adama.com/brasil/pt/adama-inovacao/adama-alvo} é um aplicativo para auxiliar na identificação de doenças, pragas e ervas daninhas em diferentes culturas. Um dos seus principais objetivos é fornecer uma base de dados offline para o usuário, de forma que seja possível analisar e comparar em tempo real uma praga encontrada em uma folha com uma foto da mesma. Além disso,  o aplicativo permite o envio de imagens por parte do usuário para serem analisadas pelos especialistas da Adama, que posterioemente podem catalogá-la e disponibilizá-la no aplicativos, auxiliando outros agricultores.

\subsection{Múltiplos usuários}

A obtenção dos dados utilizados nos sistemas computacionais de apoio à agricultura, quando inseridas de forma manual, normalmente levam em consideração apenas uma única fonte de informações, que normalmente é o próprio agricultor. Apesar dos possíveis insights coletados, a informação poderia ser enriquecida, se neste processo, houvesse por exemplo a visão de outros usuários. Por exemplo, para uma nova praga identificada em uma folha de soja, o \textit{agricultor} poderia inserir as informações relacionadas as suas experiências, como por exemplo se já ouve alguma praga com indícios semelhantes aos encontrados, a quanto tempo, por quanto tempo, e como foi resolvido. Enquanto que o \textit{engenheiro agrônomo} poderia inserir para esta mesma entrada informações extraídas de sua experiência em outras propriedades.

Desta forma consegue-se inserir diversas dimensões a partir de uma única entrada de dados.

\subsection{Dispositivos Alvos}

Segundo uma pesquisa divulgada pela FGV-SP o Brasil tem hoje dois dispositivos digitais por habitante, incluindo smartphones, computadores, notebooks e tablets. Em 2019, o País terá 420 milhões de aparelhos digitais ativos. Desta forma, fica bastante claro que estes dispositivos já fazem parte do dia a dia de muitas pessoas, inclusive dos que fazem parte de toda a cadeia Agrícola.

A utilização de smartphones e computadores, propõe uma solução acessível para um dos pontos em \cite{agriculture-40-rose}, que argumenta, que é necessário ampliar as noções de \"inclusão\" em inovação responsável para atender às diversas formas que a que os agricultores devem interagir com as fazendas inteligentes. Estes dispositivos já fazem parte da rotina de trabalho do público alvo, não sendo necessários outros investimentos ou aquisições, e ainda assim, possibilitando a inclusão deste público em novas ferramentas criadas para a Agricultura 4.0.

\section{Objetivo}

O objetivo deste trabalho é analisar as informações de entradas dos usuários como por exemplo: imagens, anotações ou vídeos, no âmbito de sistemas computacionais de suporte a decisão para a agricultura digital. Com base nesta análise, um estudo deverá ser realizados para identificar o impácto da qualidade dos resultados com relação aos métodos de entrada de dados, fazendo um comparativo entre as obtenções automáticas de dados versus a introdução manual com apoio de diretrizes de usuabilidade e a experiência do usuário sobre seu domínio.

\section{Trabalhos Relacionados}

[Ainda não trabalhados]

- Decision support systems for agriculture 4.0: Survey and challenges
- A review of visualisations in agricultural decision support systems: An HCI perspective
- Developing successful environmental decision support systems: Challenges and best practices
- AgroDSS: A decision support system for agriculture and farmin
- Considering farmers' situated knowledge of using agricultural decision support systems (AgriDSS) to Foster farming practices: The case of CropSAT 

\section{Metodologia}
	Métodos
		Revisão Sistemática da Literatura
		Avaliação - Auxílio da Embrapa
	Materiais
		Smartphones

\section{Cronograma}

4 anos
Disciplinas/Revisão da literatura no primeiro ano

\section{Resultados Esperados}

\section{Referências}

\bibliographystyle{sbc}
\bibliography{pre-project}

\end{document}
