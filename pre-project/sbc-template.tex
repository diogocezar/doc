\documentclass[12pt]{article}

\usepackage{sbc-template}
\usepackage{graphicx,url}
\usepackage[utf8]{inputenc}
\usepackage[brazil]{babel}
     
\sloppy

\title{Um estudo sobre as entradas de dados e a experiência do usuário na construção de sistemas para a Agricultura 4.0}

\author{Diogo Cezar Teixeira Batista\inst{1}, Cleber Gimenez Corrêa\inst{1}}

\address{Departamento de Computação -- Universidade Tecnológica Federal do Paraná
  (UTFPR)\\
  Cornélio Procópio -- Paraná -- Brazil
  \email{diogoctb@gmail.com, clebergimenez@utfpr.edu.br}
}

\begin{document} 

\maketitle

\begin{abstract}
  abstract
\end{abstract}
     
\begin{resumo} 
  resumo
\end{resumo}


\section{Introdução}

Serão grandes os desafios relacionados à utilização de tecnologias digitais como ferramentas de apoio ao trabalho humano de forma eficiente, eficaz e justa. Para isso, a humanidade apoia-se nas descobertas e comprovações de milhares de anos da comunidade científica, que proporcionam resultados inimagináveis, alcançados através da exploração das técnicas e métodos definidos e aprimorados ao longos dos anos.

A indústria é um exemplo de como essa evolução aconteceu. Seu marco acontece com a definição da Era Artesanal, na qual, os bens e serviços eram criados por um único indivíduo: o artesão. Este era responsável por deter todo o conhecimento dos métodos e processos para a fabricação do produto ou a prestação de um serviço. Com a criação e o aperfeiçoamento das máquinas a vapor, estes processos puderam ser minimamente automatizados, e isso deu origem a Primeira Revolução Industrial. A Segunda Revolução Industrial, acontece com a utilização da eletricidade e principalmente dos processos que possibilitaram a implantação produção em massa idealizadas por Henry Ford. Já a Terceira Revolução Industrial se inicia após a Segunda Guerra Mundial, com o descobrimento e a utilização da robótica e do uso de computadores para a automação das indústrias.

Mais recente, muito tem se falado sobre a Quarta Revolução Industrial, que surge em 2011 na Alemanha, com a proposta de oferecer para a indústria o que há de mais moderno em automação e sistemas inteligentes, possibilitando uma série de melhorias como por exemplo: a redução dos custos, a economia de energia e o aumento da segurança. Essas e outras melhorias têm sido exploradas através da utilização de ferramentas e tecnologias como: Big Data, Analytcs, Serviços de Núvem, Impressões 3D, Segurança Cibernética, Robôs Autônomos, Internet das Coisas, Sensores sem Fio, Realidade Aumentada, Simulação, Integração Horizontal e Integração Vertical.

Além da Indústria, outro setor que evoluiu muito ao longo dos anos foi a Agricultura. Esta, foi fundamental para alavancar o desenvolvimento das civilizações ao longo da história, possibilitando que comunidades nômades se estabilizassem em determinadas regiões, explorando técnicas agrícolas para a produção de seu sustento. Entretanto, o cenário atual é bem diferente dos enfrentados pelas primeiras civilizações, o crescimento populacional, e a tendência das pessoas a se tornarem cada vez mais exigentes com o que é consumido, gera um grande desafio para o setor agrícola, que deve se posicionar para atender tais demandas. Desta forma, novas técnicas vêm sido exploradas, estas, compõem o termo Agricultura 4.0, ou Agricultura Digital. Estas utilizam os mesmos métodos e processos inovadores já explorados pela Indústria 4.0, incluindo: Agricultura e Pecuária de Precisão, a automação e a robótica agrícola, além de outras técnicas como Big Data e Internet Das Coisas.

A utilização de componentes tecnológicos não é uma novidade na Agricultura, a exploração, por exemplo, de tecnologias como o GPS, estão presentes já a algum tempo na linha de produção agrícola, conhecida como Agricultura de Precisão. Já a Agricultura 4.0 expande as possibilidades de explorações tecnológicas, com a utilização por exemplo de: \textit{Sensores} que possibilitam a coleta autmática de dados sobre o solo, \textit{Drones ou Satélites} que podem apresentar recursos de imagem cada vez mais avançados, auxiliando no aumento de produtividades e ajudando a reduzir danos nas lavouras, uma vez que possibilitam o monitoramento em tempo real, \textit{SmartPhones} que proporcionam uma interface de entrada de dados rápida, acessível e já conhecida por seu usuário.

Não há dúvidas de que a tecnologia pode colocar a agricultura em seu próximo nível, A Agricultura 4.0 de aliada à produções mais variadas e a utilização de sistemas de apoio à decisões podem levar à produções mais eficientes e ao consumo mais inteligente e sustentável. Entretanto, essa transição não é uma tarefa simples. A utilização destas ferramentas tecnológicas devem mudar a forma como os agricultores interagem com suas terras, com isso, cada vez menos os trabalhadores precisão colocar a mão na massa, mudando a forma como sempre interagiam com seu ofício. E isso pode dificultar a adesão de novas tecnologias. A utilização das técnicas relacionadas a agricultura de precisão demonstraram certa resistência por parte dos agricultores. Mas o uso em larga escala de IA, robótica e outras inovações emergentes, tem o claro potencial de causar consequências sociais não intensionais, imprevistas e indesejadas. Por este motivo, é importante focar em uma peça fundamental para o sucesso da aplicação de novas tecnologias: o usuário.

Grande parte das tecnologias adotadas pela Agricultura 4.0 estão inseridas (pelo menos em algum momento) em um sistema computacional. Sabe-se que um sistema computacional é basicamente composto pela \textit{entrada de dados}, o seu \textit{processamento} e a disponibilização dos seus \textit{resultados}. As duas últimas etapas, por mais refinadas que estejam, dependem da qualidade da fonte de dados ou seja da entrada das informações.

No âmbito agrícola, por mais automatizados que possam estar os processos de obtenções dos dados, com sensores automáticos ou análise de imagens, por exemplo, temos dois principais pontos que podem aumentar ainda mais a resistência à adoção das tecnologias: \textit{Indisponibilidade de Equipamentos} ou \textit{Conhecimento Empírico}. No cenário atual, não são todos os agricultores que podem destinar parte do seu faturamento para a aquisição de equipamentos modernos que facilitariam a coleção de dados. Além disso, existe uma variável de grande importância que deve ser inserida na equação da Agricultura 4.0, a experiência do agricultor com a sua propriedade.

Por este motivo, a idéia deste trabalho é proporcionar um estudo direcionado aos métodos de obtenção dos dados, considerando técnidas de interface e usabilidade, para propor e experimentar diretrizes que apoiem o ususário (agricultor) a através de interfaces com dispositivos computacionais, apliquem entradas com base em informações provenientes de sua própria experiência.

\section{Objetivo}

O objetivo é analisar as informações de entradas dos usuários (imagens, anotações, vídeos etc) no âmbito de sistemas computacionais de suporte a decisão para a agricultura digital, considerando a necessidade dessas entradas, as interfaces para essas entradas, a relação dessas entradas com outras informações (imagens, temperatura, umidade, biomassa), provenientes de sensores, como dispositivos posicionados na plantação, imagens de satélites, veículos não tripulados, etc.

\section{Trabalhos Relacionados}

- Decision support systems for agriculture 4.0: Survey and challenges
- A review of visualisations in agricultural decision support systems: An HCI perspective
Os dois artigos acima são de revisão, que contêm os artigos da área já levantados e organizados.
- Developing successful environmental decision support systems: Challenges and best practices (Não tive acesso)
- AgroDSS: A decision support system for agriculture and farming (Não tive acesso)
- Considering farmers' situated knowledge of using agricultural decision support systems (AgriDSS) to Foster farming practices: The case of CropSAT (Não tive acesso)

\section{Metodologia}

	Métodos
		Revisão Sistemática da Literatura
		Avaliação - Auxílio da Embrapa

	Materiais
		Smartphones

\section{Cronograma}

4 anos
Disciplinas/Revisão da literatura no primeiro ano

\section{Resultados Esperados}

\section{Referências}

Bibliographic references must be unambiguous and uniform.  We recommend giving
the author names references in brackets, e.g. \cite{knuth:84},
\cite{boulic:91}, and \cite{smith:99}.

The references must be listed using 12 point font size, with 6 points of space
before each reference. The first line of each reference should not be
indented, while the subsequent should be indented by 0.5 cm.

\bibliographystyle{sbc}
\bibliography{sbc-template}

\end{document}
