\documentclass[12pt]{article}
\usepackage{class/sbc-template}
\usepackage{graphicx,url}
\usepackage[utf8]{inputenc}
\usepackage[brazil]{babel}
\usepackage{multirow}
\usepackage{hyperref} 
\usepackage{abntex2cite}
\usepackage{color}

\bibliographystyle{sbc}
     
\sloppy

\title{Método de Engenharia de Software para o desenvolvimento de sistemas de monitoramento de plantações e de colaboração entre usuários no contexto da agricultura 4.0}

\author{Diogo C. T. Batista\inst{1}, Cléber G. Corrêa\inst{2}, Letícia M. Peres\inst{1}, Roberto Pereira\inst{1}}

\address{Universidade Federal do Paraná (UFPR)\\
	Curitiba -- Paraná -- Brasil
	\nextinstitute
	Universidade Tecnológica Federal do Paraná (UTFPR)\\
  Cornélio Procópio -- Paraná -- Brasil
	\email{diogo@diogocezar.com,clebergimenez@utfpr.edu.br,\{lmperes,rpereira\}@inf.ufpr.br}
	}


\begin{document} 

\maketitle
     
\begin{resumo} 
A agricultura 4.0 explora a utilização das mais recentes tecnologias computacionais, envolvendo a agricultura e pecuária de precisão, bem como a agricultura digital, que empregam a automação, a robótica agrícola, \textit{big data}, a Internet das Coisas, entre outras. A utilização dessas tecnologias busca uma produção agrícola eficiente e sustentável, possibilitando a tomada de decisão. Por exemplo, o monitoramento da umidade e da fertilidade do solo para a economia de água na irrigação e no uso de insumos na adubagem dos solos. Entretanto, o acesso aos recursos necessários, como sensores, para a exploração dessas tecnologias não é a realidade de grande parte do setor agrícola. Adicionalmente, a resistência na adoção de novas tecnologias é um problema ainda em aberto. Este projeto propõe um método, técnica ou ferramenta de Engenharia de Software para apoiar o desenvolvimento de sistemas computacionais no contexto da agricultura 4.0, que permita a colaboração entre os usuários e o monitoramento da plantação, para apoiar a tomada de decisão. \\

\textbf{Palavras-chave:} agricultura 4.0; Engenharia de Software; monitoramento agrícola; sistemas colaborativos; sistemas de apoio a decisão.
\end{resumo}

\section{Introdução}
\label{sec:introducao}

São grandes os desafios relacionados à utilização de tecnologias digitais como ferramentas de apoio ao trabalho humano de forma eficiente e eficaz \cite{Rose:2019}. Para isso, a humanidade apoia-se nas descobertas que proporcionam resultados, alcançados por meio da exploração de métodos, técnicas e ferramentas.

Muito tem se falado sobre a Quarta Revolução Industrial, que surge em 2011 na Alemanha, com a proposta de oferecer para a indústria o que há de mais moderno em automação e sistemas inteligentes, possibilitando uma série de melhorias como a redução dos custos, a economia de energia e o aumento da segurança. Essas e outras melhorias têm sido exploradas por meio da utilização de ferramentas e tecnologias computacionais, tais como \textit{big data}, \textit{analytics}, serviços de nuvem, impressões 3D, segurança cibernética, robôs autônomos, Internet das Coisas, sensores sem fio, realidade aumentada, simulações, integração horizontal e integração vertical \cite{Souza:2017}.

Além da indústria, outro setor que evoluiu muito ao longo dos anos foi a agricultura. Esta foi fundamental para alavancar o desenvolvimento das civilizações ao longo da história, possibilitando que comunidades nômades se estabilizassem em determinadas regiões, explorando técnicas agrícolas para a produção de seu sustento. Entretanto, o cenário atual é bem diferente dos enfrentados pelas primeiras civilizações, o crescimento populacional gera um grande desafio para o setor agrícola, que precisa se posicionar eficientemente para atender essas demandas de forma sustentável. Dessa forma, métodos tradicionais de trabalho na agricultura, como por exemplo: plantio, fertilização ou colheita, têm sido adotadas, em conjunto com novas tecnologias, incluindo: a automação, a robótica, \textit{big data} e Internet das Coisas \cite{Ribeiro:2018}.

A utilização de componentes tecnológicos não é uma novidade na agricultura, a exploração, por exemplo, de tecnologias como o \textit{Global Positioning System} (GPS), estão presentes há algum tempo na linha de produção agrícola, conhecida como agricultura de precisão. A agricultura 4.0 expande as possibilidades de explorações tecnológicas, com a utilização por exemplo de: sensores; que possibilitam a coleta automática de dados sobre o solo e clima; Veículos Aéreos Não Tripulados (VANTs ou \textit{drones}) ou satélites, que podem apresentar recursos de imagem cada vez mais avançados; \textit{smartphones}, que proporcionam uma interface de entrada e saída de dados rápida, acessível e conhecida por usuários de dispositivos eletrônicos \cite{Shepherd:2018}. 

Essas tecnologias permitem o monitoramento das plantações para apoiar a tomada de decisão, levando a uma produção eficiente, inteligente e sustentável. Entretanto, essa transição não é uma tarefa trivial. Segundo \citeonline{Rose:2019}, a utilização destas ferramentas tecnológicas devem mudar a forma como os agricultores interagem com suas plantações, com isso, cada vez menos os trabalhadores precisarão atuar em atividades manuais, mudando a forma de como sempre interagiram com seu ofício. E isso pode dificultar a adesão a novas tecnologias. A utilização das técnicas relacionadas a agricultura de precisão demonstraram certa resistência por parte dos agricultores \cite{Rose:2019}. 

Adicionalmente, no cenário atual, não são todos os agricultores que podem destinar parte do seu faturamento para a aquisição de equipamentos (quantidade e qualidade), como sensores, que facilitariam a coleta de dados e o monitoramento da plantação. E existe também uma variável de grande importância que deve ser inserida na equação da agricultura 4.0, a experiência do agricultor ou usuário com a sua propriedade e a sua área de atuação, possibilitando a colaboração entre usuários para a divisão de tarefas e controle de um sistema computacional complexo. Segundo \citeonline{Rose:2019} aproximar o usuário final do processo do desenvolvimento de sistemas para a agricultura 4.0, pode ser uma estratégia eficiente para a adoção de novas tecnologias.

Finalmente, a Engenharia de Software estuda métodos, técnicas e ferramentas que auxiliam no desenvolvimento de sistemas computacionais \cite{Sommerville:2011}. Dessa forma, o presente trabalho consiste na proposta de um método, técnica ou ferramenta para o desenvolvimento de sistemas computacionais no contexto da agricultura 4.0, que permita a colaboração entre os usuários e o monitoramento da plantação para a tomada de decisão.

\section{Revisão da literatura}
\label{sec:revisao_literatura}

Segundo \citeonline{Sprague:1991}: ``Um sistema de apoio à decisão (SAD) é um sistema de informação que apoia qualquer processo de tomada de decisão em áreas de planejamento estratégico, controle gerencial e controle operacional''. Na definição de \citeonline{Henry:1990}: ``SAD é um sistema baseado em computador que auxilia o processo de tomada de decisão utilizando dados e modelos para resolver problemas não estruturados''. Os SAD são sistemas que auxiliam no processo de tomada de decisão, não somente fornecendo informações para o apoio à decisão, mas, também, analisando alternativas e propondo soluções; podendo ainda usar o histórico de tomada de decisões para simular situações e inferir resultados mais refinados. Este processo de tomada de decisão envolve a interação constante do usuário com um ambiente de apoio à decisão, criado para orientar as decisões a serem tomadas. Como características destes sistemas, pode-se destacar: possibilidade de desenvolvimento rápido; facilidade em acoplar novas ferramentas de apoio a decisão; flexibilidade na manipulação das informações \cite{John:1989}; individualização e orientação para o usuário que toma as decisões (com flexibilidade de adaptação ao estilo pessoal) \cite{Mittra:1986}; usabilidade, permitindo que o usuário entende, use o modifique de forma fácil e interativa \cite{Awad:1988}.

Segundo \citeonline{Frank:2014} designers e desenvolvedores frequentemente cometem erros, assumindo que conhecem a forma de como uma tecnologia será utilizada pelo usuário do sistema ou imaginando que todos usuários são iguais. Obviamente, usuários têm percepções e necessidades próprias, e por isso é importante compreendê-los e analisá-los. Argumenta-se ainda que, investindo nessa etapa de entendimento do usuário, consegue-se benefícios como: produtos com maior índice de usabilidade; economia financeira; sistemas mais seguros. Mas para isso, existem algumas dificuldades no processo, como por exemplo: somente compreender o usuário não garante o sucesso do produto e entender quando parar de analisar o usuário pode ser difícil.

Segundo \citeonline{Ellis:1991} os sistemas colaborativos (\textit{groupware}) oferecem uma interface compartilhada entre dois ou mais usuários, para a realização de uma tarefa em comum. Um de seus objetivos é auxiliar na cooperação, comunicação e na coordenação das atividades a serem realizadas nos sistemas computacionais. \citeonline{Roth:2002} argumenta que aplicações colaborativas permitem que pessoas distribuídas geograficamente possam realizar uma tarefa de forma colaborativa, sem que exista um atraso significativo na comunicação. Como principais características dos sistemas colaborativos, destaca-se:

\begin{itemize}
	\item Percepção: busca fornecer a constante notificação sobre as possíveis modificações realizadas pelos usuários; isso torna as interações consistentes;
	\item Cooperação: segundo \cite{Fuks:2003} ``cooperação é a operação conjunta dos membros do grupo no espaço compartilhado visando a realização das tarefas gerenciadas pela coordenação'';
	\item Coordenação: é necessária para que as tarefas sejam realizadas com sucesso. \citeonline{Ellis:1991} destaca algumas formas de manter o sistema coordenado para o usuário: permitir a visualização das suas ações, assim como as ações de outros usuários ou participantes; gerar avisos e alertas automáticos, entre outros. Sem a coordenação o sistema pode ficar suscetível ao conflito de atividades ou da realização de ações repetidas;
	\item Comunicação: consiste na troca de mensagens entre os usuários do sistema. Segundo \cite{Fuks:2003} a comunicação ``dá suporte às interações entre os participantes, podendo gerenciar as transições de estados, os eventos de diálogo e os compromissos de cada participante''
\end{itemize}

No cenário agrícola, a utilização de SAD é amplamente empregado para auxiliar os usuários na tomada de decisões estratégias, que ajudam no cultivo de suas plantações, como demonstrado na seção \ref{sec:trabalhos_relacionados}. Os diversos usuários que participam deste domínio possuem suas particularidades, e por isso, o estudo e entendimento de suas \textit{personas} é importante para aproximar e engajar a utilização do sistema. Por fim, a utilização de sistemas colaborativos agrícolas pode ser um cenário promissor pela diversidade de usuários, como agricultores, técnicos agrícolas, engenheiros agrônomos, entre outros, que podem atuar em conjunto para auxiliar na tomada de decisão.

\section{Trabalhos Relacionados}
\label{sec:trabalhos_relacionados}

\citeonline{Gutierrez:2019} demonstram por meio da comparação de trabalhos, que a utilização de diversas fontes de informações de entradas é um ponto desejável para a automatização e autonomia na agricultura 4.0. As principais comparações estão relacionadas às formas de visualização de dados processados nesse contexto. As entradas dos usuários são tratadas de forma superficial, bem como a colaboração entre esses usuários. Os autores mencionam a preocupação na inserção de diversas dimensões em uma mesma fonte de dados, e que o processo de obtenção de dados refinados pode aprimorar os resultados obtidos.

O trabalho de \citeonline{Zhai:2020} apresenta como desafios para os sistemas de apoio a decisão agrícola: a simplificação de interfaces gráficas dos usuários e utilização de conhecimento dos especialistas do domínio do sistema. Além disso, são abordadas alternativas às entradas de dados comuns, como por exemplo: a utilização do reconhecimento de voz ou a utilização gestos. Outro ponto importante está relacionado ao processo de extração das informações da fonte de dados. Modelos tridimensionais, análise de imagens ou vídeos, e imagens interativas podem ser pontos ainda a serem explorados. Para completar, a colaboração entre usuários também é pouco explorada.

Nos trabalhos de \citeonline{Walling:2020} e \citeonline{Lundstrom:2018} é mostrada a importância da participação do usuário no processo de obtenção de dados em sistemas relacionados a agricultura 4.0. Profissionais com a experiência de seu domínio, são capazes de enriquecer toda a cadeia com a inferência de seu conhecimento nos sistemas de apoio a decisão. Com esses estudos fica claro que a participação multidisciplinar em toda a cadeia dos sistemas de apoio a decisão podem fazer uma diferença expressiva nos resultados.

No trabalho de \citeonline{Mudisshu:2016} argumenta-se sobre a diminuição da produção agrícola ano a ano no Japão, causada pela liberação das importações de produtos agrícolas. O projeto tem o objetivo de reduzir os custos nas aquisições de dados meteorológicos, com o desenvolvimento de um aplicativo para \textit{smartphone} para a visualização de dados.

No trabalho de \citeonline{Devitt:2017} é descrito um aplicativo móvel para o apoio a tomada de decisões com relação ao manejo de plantas daninhas. A ferramenta interativa auxilia os produtores de algodão na exploração do impacto causado por uma praga, além de traçar sugestões de estratégias individuais. A pesquisa aborda o desafio de envolver as partes interessadas na tomada de decisão de três maneiras: 1) reconhecendo as prioridades cognitivas individuais; 2) visualizando o manejo científico de plantas daninhas com uma interface móvel atraente e 3) representando incertezas na decisão e o risco ponderado em relação às prioridades cognitivas.

No trabalho de \citeonline{Fallon:2018} argumenta-se a importância de previsões climáticas sazonais para apoiar as decisões agrícolas de curto prazo, e planos de adaptação climática de longo prazo. Foi desenvolvida uma ferramenta de gerenciamento de terras (LMTool), um protótipo de serviço climático sazonal para gestores fundiários. Esta ferramenta fornece previsões de (1 a 3 meses à frente) para agricultores no sudoeste do Reino Unido, além de previsões meteorológicas específicas de 14 dias durante os meses de inverno, quando a habilidade das previsões sazonais é maior. Descreve-se ainda, os processos de desenvolvimento do sistema em conjunto com os agricultores.

O trabalho de \citeonline{Luvisi:2011} utiliza um método para combinar dados armazenados em um sistema de RFID (\textit{Radio-Frequency IDentification}), implantado dentro de plantas de videira e um GPS. O \textit{software} GIS foi usado para registrar coordenadas geográficas dos pontos detectados e para alimentar um banco de dados específico no qual as informações úteis para a fase de posicionamento sejam armazenadas. O produto final é um mapa digital acessível por dispositivos móveis ou uma aplicação \textit{desktop} que representa o ``vinhedo virtual''. Nessa representação digital, cada planta de videira marcada por RFID pode ser selecionada, visualizada e editada.

No trabalho de \citeonline{Tan:2012} argumenta-se que os operadores de campo, podem coletar dados volumosos que fornecem em tempo real dados que representam as condições do campo em determinado momento. Destaca-se ainda, que é um grande desafio analisar todos esses dados de forma eficiente e utilizá-los para melhorar as decisões agrícolas. A proposta é um \textit{software} para análise e visualização de dados com precisão, que conta com três características distintas: 1) um meta-modelo de componente de importação, capaz de importar dados em vários formatos a partir de uma variedade de dispositivos em diferentes configurações; 2) um processamento de dados orientado por fluxos de trabalho, no qual um usuário pode definir seus fluxos de trabalho e adicionar processamento de dados definido de forma personalizada para uma aplicação específica; 3) uma arquitetura seguindo um modelo cliente-servidor que com suporte a uma variedade de dispositivos clientes, incluindo dispositivos móveis.

Por fim, no trabalho de \citeonline{Zheng:2017} destaca-se a demanda por aplicativos móveis como ferramentas de apresentação de dados. O trabalho busca projetar e desenvolver estratégias para uma apresentação mais amigável dos dados para seus usuários.

Percebe-se nos trabalhos levantados abordagens de desenvolvimento, sem abordar detalhes sobre a aplicação de métodos, técnicas e ferramentas da Engenharia de Software. A participação dos usuários no desenvolvimento, a implementação de interfaces humano-computador para entrada e saída de informações, e a necessidade de diversos dispositivos para monitoramento de diferentes características das plantações, são pontos mencionados com recorrência.

\section{Justificativa}
\label{sec:justificativa}

O presente projeto visa contribuir com determinados \textit{objetivos de desenvolvimento sustentável}, descritos pela Organização das Nações Unidas \cite{ONU:2020}. No documento da organização, são descritos 17 objetivos que buscam concretizar os direitos humanos e promover a igualdade. Para isso, tais objetivos e metas foram definidos com o intuito de estimular ações para os próximos 15 anos, em áreas de importância crucial para a humanidade e para o planeta. Dentre os objetivos, dois deles têm relação direta com este trabalho:

\begin{itemize}
	\item \textit{Objetivo 2: fome zero e agricultura sustentável} - Aumentar a eficiência da produção agrícola é consequência direta da proposta deste trabalho;
	\item \textit{Objetivo 12: consumo e produções sustentáveis} - Neste quesito, o trabalho contribui com a proposta de uma solução eficiente, acessível (utilizando um \textit{smartphone} como equipamento, por exemplo) e sem a necessidade da aquisição de novos equipamentos para integrações tecnológicas.
\end{itemize}

No cenário agrícola, são vários os papéis dos profissionais que compõem toda a cadeia de produção. A obtenção dos dados utilizados nos sistemas computacionais de apoio à agricultura deve levar em consideração a diversidade de usuários, como agricultores, técnicos agrícolas, engenheiros agrônomos, entre outros. As experiências desses múltiplos profissionais poderiam ser utilizadas em sistemas computacionais para auxiliar na tomada de decisão. Por exemplo, para uma nova praga identificada em uma folha de soja, o \textbf{agricultor} poderia inserir as informações relacionadas as suas experiências, como por exemplo, se já houve algum inseto com indícios semelhantes aos encontrados, a quanto tempo, por quanto tempo, e como foi resolvido. Enquanto isso o \textbf{engenheiro agrônomo} poderia inserir para esta mesma entrada informações extraídas de sua experiência em outras fazendas, \textbf{colaborando} para solucionar o problema de identificação dessa praga. Dessa forma consegue-se extrair diversas \textbf{dimensões} a partir de uma única entrada de dados \cite{Walling:2020}.

Outro ponto importante é a adesão a novas tecnologias. Segundo uma pesquisa divulgada pela Fundação Getúlio Vargas (FGV-SP), o Brasil tem hoje dois dispositivos digitais por habitante, incluindo \textit{smartphones}, computadores, \textit{notebooks} e \textit{tablets} \cite{FGV:2020}. Nesse sentido, o país teria aproximadamente 420 milhões de aparelhos digitais ativos. Dessa forma, fica bastante claro que estes dispositivos já fazem parte do dia a dia de muitas pessoas, inclusive dos que fazem parte de toda a cadeia agrícola, sejam agricultores de pequeno e médio portes.

A utilização de \textit{smartphones} configura uma solução acessível. \citeonline{Rose:2019} argumentam que é necessário ampliar as noções de ``inclusão'' de forma responsável, para atender às diversas maneiras de como os agricultores devem interagir com as fazendas inteligentes. Estes dispositivos já fazem parte da rotina de trabalho do público alvo, não sendo necessários outros investimentos ou aquisições, e ainda assim, possibilitando a inclusão desse público em novas ferramentas criadas para a agricultura 4.0.

No cenário proposto pela agricultura 4.0, grande parte dos trabalhos evidenciam que as informações que alimentam os sistemas, podem ser obtidas de forma automatizada, principalmente por meio de sensores. Entretanto, nem sempre essa é uma realidade aplicável a todos, pelo alto custo (aquisição e manutenção) e difícil aderência.

\section{Objetivos}
\label{sec:objetivos}

Na presente seção são descritos o objetivo geral e os objetivos específicos.

\subsection{Objetivo geral}
\label{subsec:objetivo_geral}

O objetivo do presente trabalho trata de uma proposta de método, técnica ou ferramenta para apoiar o desenvolvimento de sistemas computacionais no contexto da agricultura 4.0, que permita a colaboração entre os usuários e o monitoramento da plantação, para apoiar a tomada de decisão.

\subsection{Objetivos específicos}
\label{subsec:objetivos_especificos}

\begin{itemize}
	\item Pesquisar métodos, técnicas e ferramentas de Engenharia de Software para o desenvolvimento de sistemas desse tipo;
	\item Levantar sistemas de apoio à tomada de decisão e colaborativos, em especial os direcionados para monitoramento de plantações e agricultura 4.0, analisando seus pontos positivos e negativos;
	\item Propor método, técnica ou ferramenta para o desenvolvimento de sistemas desse tipo;
	\item Validar a proposta em determinados cenários para levantar benefícios e limitações.
\end{itemize}

\section{Problema a ser Investigado}
\label{sec:problema_investigado}

O problema a ser investigado é como desenvolver sistemas computacionais de monitoramento e de colaboração entre usuários no contexto da agricultura 4.0 para a tomada de decisão. Aparentemente é inviável para a maioria dos produtores rurais, especialmente os pequenos e médios agricultores, adotar um sistema completo para a agricultura 4.0 em um primeiro momento \cite{Rose:2019}, devido as limitações tecnológicas (conexão com a Internet), altos custos de aquisição e manutenção de equipamentos (drones e sensores), dificuldades no uso de interfaces humano-computador pelos usuários.

No âmbito agrícola, é bastante comum grandes áreas de plantio. Mapear completamente uma área com dispositivos automatizados poderia gerar um custo muito grande para os agricultores, principalmente com a necessidade de manutenção dos equipamentos que captam as informações dos campos, bem como o consumo de energia elétrica por estes dispositivos.

Dessa forma, a Engenharia de Software pode fornecer métodos, técnicas e ferramentas para o desenvolvimento gradual e adaptativo de sistemas computacionais de monitoramento e colaboração na agricultura 4.0 para apoiar a tomada de decisão. Gradual porque pode ser por etapas, conforme a cobertura de internet avança para regiões afastadas na área agrícola, conforme sensores ou outros equipamentos são adquiridos, conforme os usuários aprendem a trabalhar com sistemas computacionais. Adaptativo porque equipamentos que não podem ser adquiridos em uma determinada etapa podem ser substituídos por dispositivos disponíveis, como os \textit{smartphones}, que possuem recursos, tais como interface gráfica para inserção e visualização de informações, câmera, conexão com a Internet.

A utilização de recursos automatizados, como drones ou sensores, tendem a facilitar muito o processo de obtenção de dados em uma fazenda que aplica técnicas da agricultura 4.0. Tratando de cenários nos quais essa já seja uma realidade, estes recursos automatizados normalmente apenas armazenam uma representação da realidade obtida em algum instante. Por exemplo, um sensor de umidade, poderia registrar os índices de umidade do solo a cada hora. Ou ainda, um drone, poderia fotografar uma área com baixa produtividade. Estes dados não levam em consideração alguns aspectos que poderiam ser enriquecidos com a arguição de um especialista, por exemplo um agrônomo. A inclusão de novas dimensões na coleta de dados, unindo por exemplo, a obtenção automática da umidade do solo com informações e experiências de agrônomos especialistas, poderiam, por exemplo, informar que apesar de uma leitura de umidade ter ficado fora da expectativa, isso foi advento de uma anormalidade, por ele observada. Como há mais de um especialista em uma mesma tarefa, regras de colaboração devem ser especificadas.

Para os cenários nos quais se utiliza a captação automática dos dados, grande parte das abordagens vistas em \cite{Massruha:2017} fazem essa captura de forma contínua e interruptamente, propondo abordagens que tratem grandes volumes de dados. A disponibilidade das informações de monitoramento é uma questão a ser levada em consideração.

A experiência do usuário com um dispositivo computacional pode influenciar diretamente na adoção de novas tecnologias ou sistemas. É comum a resistência por parte de novos usuários na adoção de sistemas com que ainda não tiverem contato. Desse modo, são relevantes descobertas relacionadas ao entendimento do perfil do público alvo, representando os usuários e suas peculiaridades para criar interfaces que possam ser objetivas, claras, atrativas e principalmente funcionais. Outro desafio a ser explorado é a interação multidimensional da entrada e análise de dados, que propõe-se ser refinada, simultaneamente, por diversos colaboradores e que resultem em uma única fonte de dados, possibilitando a inferência de apoio a decisão extraídas de dados refinados e alinhados com as realidades do cenário agrícola apontado por parte da cadeia que já atua na produção agrícola. A dependência do usuário também deve ser levada em conta.

Para definir o método, técnicas podem ser utilizadas para definir e implementar camadas (elementos, relações internas e externas), bem como ferramentas, para permitir a visualização da arquitetura de camadas. As camadas podem ser definidas de acordo com os tipos de informação (umidade do solo, fertilidade do solo, temperatura do ambiente, chuvas, pragas, ervas daninhas, doenças, folha da planta, altura da planta, densidade da vegetação), contendo o número e a disposição dos sensores e outros equipamentos, as relações entre sensores de uma camada e entre as camadas, os sensores virtuais (substituídos ou adaptados).

\section{Metodologia}
\label{sec:metodologia}

Os métodos para a realização da pesquisa incluem:

\begin{enumerate}
	\item Revisão da literatura e análise crítica sobre sistemas de monitoramento e colaborativos para a área agrícola, buscando explorar os pontos positivos e negativos dos trabalhos estudados;
	\item Definição dos perfis de usuários, em caso de necessidade. Nesse ponto, destaca-se a importância do trabalho entre a UTFPR e a Embrapa \footnote{Empresa Brasileira de Pesquisa Agropecuária, situada em Londrina, trabalha principalmente com a produção da soja. A Embrapa é uma empresa pública de pesquisa vinculada ao ministério da agricultura, pecuária e abastecimento do Brasil. O seu principal objetivo é a viabilização de soluções provenientes de pesquisas, além do desenvolvimento para a sustentabilidade da agricultura em benefício da sociedade brasileira. Endereço eletrônico: https://www.embrapa.br/};
	\item Planejamento, organização e execução de experimentos envolvendo usuários, em caso de necessidade. Estudo da utilização de questionários para coletar opiniões referentes as percepções sobre a utilização do sistema e atividades de colaboração. Outras abordagens para obtenção de resultados podem ser empregadas como:
	\begin{itemize}
		\item Testes A/B;
		\item Análise de Abandono;
		\item Avaliação do Aplicativo na Loja;
		\item Análise de Usuários Ativos do Aplicativo.
	\end{itemize}
	\item Análise de resultados utilizando testes estatísticos para verificar a significância das diferenças entre grupos amostrais comparados (Friedman, ANOVA – Análise da Variância, t-test, Wilcoxon, Mann-Whitnney etc), podendo englobar dados de testes com usuários e obtidos por meio de sensores; e estatística descritiva (principalmente gráficos). Os testes estatísticos dependem das características dos dados (normalidade da distribuição, homogeneidade das variâncias, independência e aleatoriedade na coleta etc) e oferecem intervalos de confiança para a análise.
\end{enumerate}

\section{Atividades}
\label{sec:atividades}

O plano de trabalho é composto pelas seguintes atividades principais, dispostas no cronograma apresentado na Tabela \ref{tab:cronograma}, separadas por quadrimestres de cada ano:

\begin{enumerate}
	\item Obtenção de créditos, cursando disciplinas;
	\item Estudo de métodos para mapeamento ou revisão sistemática, permitindo o levantamento de trabalhos na literatura;
	\item Estudo do estado da arte, considerando métodos, técnicas ou ferramentas para projetar, implementar e avaliar sistemas computacionais colaborativos e de monitoramento voltados para a agricultura 4.0;
	\item Exame de proficiência de língua estrangeira;
	\item Criação de uma proposta inicial conceitual para o desenvolvimento de protótipos de sistemas computacionais colaborativos e de monitoramento com foco na agricultura 4.0. Especificação do método, técnica ou ferramenta para o desenvolvimento, detalhando as tarefas necessárias;
	\item Especificação e implementação de protótipos para validação, com a aplicação do método, técnica ou ferramenta proposta;
	\item Especificação de cenários para validação da proposta em conjunto com a equipe da Embrapa para o desenvolvimento de sistemas de monitoramento da plantação e de colaboração entre usuários no âmbito da agricultura 4.0;
	\item Validação da aplicação do método, técnica ou ferramenta. Testes envolvendo usuários podem ser necessários, especialmente para a avaliação do sistema computacional;
	\item Redação de artigos científicos a serem submetidos aos principais eventos e periódicos da área de interesse;
	\item Redação da tese e defesa.
\end{enumerate}

\section{Resultados Esperados}
\label{sec:resultados_esperados}

Pretende-se criar uma proposta baseada em Engenharia de Software para o desenvolvimento de sistemas computacionais no contexto da agricultura 4.0, que permita a colaboração entre os usuários e o monitoramento da plantação para apoiar a tomada de decisão. A proposta trata de um método, técnica ou ferramenta para o desenvolvimento de software nesse contexto.

Espera-se, por fim, a submissão de artigos para conferências e periódicos científicos da área de Ciência da Computação para publicação, como o \textit{ACM Computing Surveys}, \textit{Computer Graphics and Applications}, \textit{Journal on Interactive Systems}, \textit{Computers and Electronics in Agriculture}, \textit{Symposium on Applied Computing}, Simpósio Brasileiro de Engenharia de Software.

\section{Cronograma}
\label{sec:cronograma}

As atividades e os períodos serão ajustados conforme o cronograma do Programa de Pós-Graduação em Informática da UFPR e o andamento do trabalho.

\begin{table}[htbp]
	\centering
		\begin{tabular}{|c|c|c|c|c|c|c|c|c|c|c|c|c|}
			\hline
			\multirow{2}{*}{AT} & \multicolumn{3}{c|}{\textbf{Ano 1}} & \multicolumn{3}{c|}{\textbf{Ano 2}} & \multicolumn{3}{c|}{\textbf{Ano 3}} & \multicolumn{3}{c|}{\textbf{Ano 4}} \\ \cline{2-13} 
													& Q1         & Q2         & Q3        & Q1         & Q2         & Q3        & Q1         & Q2         & Q3        & Q1         & Q2         & Q3        \\ \hline
			1                   & •          & •          & •         &            &            &           &            &            &           &            &            &           \\ \hline
			2                   & •          & •          &           &            &            &           &            &            &           &            &            &           \\ \hline
			3                   & •          & •          & •         & •          & •          & •         & •          & •          & •         & •          & •          & •         \\ \hline
			4                   &            &            &           & •          &            &           &            &            &           &            &            &           \\ \hline
			5                   &            &            &           & •          & •          &           &            &            &           &            &            &           \\ \hline
			6                   &            &            &           &            & •          & •         & •          &            &           &            &            &           \\ \hline
			7                   &            &            &           &            &            & •         & •          & •          &           &            &            &           \\ \hline
			8                   &            &            &           &            &            &           &            & •          & •         & •          &            &           \\ \hline
			9                   &            & •          & •         & •          & •          & •         & •          & •          & •         & •          & •          & •         \\ \hline
			10                  &            &            &           &            &            &           &            &            &           &            & •          & •         \\ \hline
		\end{tabular}
	\caption{Cronograma de Execução das Atividades}
	\label{tab:cronograma}
\end{table}


\bibliography{project}

\end{document}