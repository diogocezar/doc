\documentclass[12pt]{article}
\usepackage{class/sbc-template}
\usepackage{graphicx,url}
\usepackage[utf8]{inputenc}
\usepackage[brazil]{babel}
\usepackage{multirow}
\usepackage{hyperref} 
\usepackage{abntex2cite}

\bibliographystyle{sbc}
     
\sloppy

\title{Um estudo sobre sistemas colaborativos eficientes para os usuários no contexto da agricultura 4.0}

\author{Diogo C. T. Batista\inst{1}, Cléber G. Corrêa\inst{1}, Letícia M. Peres\inst{2}, Roberto Pereira\inst{2}}

\address{Universidade Tecnológica Federal do Paraná (UTFPR)\\
  Cornélio Procópio -- Paraná -- Brasil
	\nextinstitute
	Universidade Federal do Paraná (UFPR)\\
	Curitiba -- Paraná -- Brasil
	\email{diogo@diogocezar.com,clebergimenez@utfpr.edu.br,\{lmperes,rpereira\}@inf.ufpr.br}
	}


\begin{document} 

\maketitle
     
\begin{resumo} 
A agricultura 4.0 explora a utilização das mais atuais tecnologias computacionais envolvendo a agricultura e pecuária de precisão, bem como a agricultura digital, que empregam a automação, a robótica agrícola, \textit{big data}, a Internet das Coisas, entre outras. A utilização dessas tecnologias busca uma produção agrícola eficiente e sustentável, possibilitando, por exemplo, a economia de água na irrigação ou de insumos na adubagem dos solos. Entretanto, o acesso aos recursos necessários para a exploração dessas tecnologias não é a realidade de grande parte do setor agrícola. Adicionalmente, a resistência na adoção de novas tecnologias é um problema ainda em aberto. Este trabalho busca a exploração de métodos ou técnicas e ferramentas, que no contexto da engenharia de software, direcionem soluções computacionais inseridas no contexto da agricultura 4.0, com o intuito de apoiar o desenvolvimento de sistemas colaborativos eficientes na realização de tarefas por usuários do setor agrícola.\\

\textbf{Palavras-chave:} agricultura 4.0; sistemas colaborativos; sistemas de apoio a decisão.
\end{resumo}

\section{Introdução}
\label{sec:introducao}

São grandes os desafios relacionados à utilização de tecnologias digitais como ferramentas de apoio ao trabalho humano de forma eficiente, eficaz e justa \cite{Rose:2019}. Para isso, a humanidade apoia-se nas descobertas que proporcionam resultados relevantes, alcançados por meio da exploração das técnicas e métodos.

Muito tem se falado sobre a Quarta Revolução Industrial, que surge em 2011 na Alemanha, com a proposta de oferecer para a indústria o que há de mais moderno em automação e sistemas inteligentes, possibilitando uma série de melhorias como a redução dos custos, a economia de energia e o aumento da segurança. Essas e outras melhorias têm sido exploradas por meio da utilização de ferramentas e tecnologias como \textit{big data}, \textit{analytics}, serviços de nuvem, impressões 3D, segurança cibernética, robôs autônomos, internet das coisas, sensores sem fio, realidade aumentada, simulações, integração horizontal e integração vertical \cite{Souza:2017}.

Além da indústria, outro setor que evoluiu muito ao longo dos anos foi a agricultura. Esta foi fundamental para alavancar o desenvolvimento das civilizações ao longo da história, possibilitando que comunidades nômades se estabilizassem em determinadas regiões, explorando técnicas agrícolas para a produção de seu sustento. Entretanto, o cenário atual é bem diferente dos enfrentados pelas primeiras civilizações, o crescimento populacional gera um grande desafio para o setor agrícola, que precisa se posicionar eficientemente para atender essas demandas de forma sustentável. Dessa forma, métodos tradicionais de trabalho na agricultura, como por exemplo: plantio, fertilização ou colheita; têm sido adotadas, em conjunto com métodos e processos inovadores incluindo: a automação, a robótica, \textit{big data} e internet das coisas \cite{Ribeiro:2018}.

A utilização de componentes tecnológicos não é uma novidade na agricultura, a exploração, por exemplo, de tecnologias como o \textit{Global Positioning System} (GPS), estão presentes há algum tempo na linha de produção agrícola, conhecida como agricultura de precisão. A agricultura 4.0 expande as possibilidades de explorações tecnológicas, com a utilização por exemplo de: sensores; que possibilitam a coleta automática de dados sobre o solo e clima; Veículos Aéreos Não Tripulados (VANTs ou \textit{drones}) ou satélites, que podem apresentar recursos de imagem cada vez mais avançados; smartphones, que proporcionam uma interface de entrada e saída de dados rápida, acessível e conhecida por usuários de dispositivos eletrônicos \cite{Shepherd:2018}.

A agricultura 4.0, com a utilização de sistemas de apoio a decisão, pode levar a produção mais eficientes, inteligente e sustentável. Entretanto, essa transição não é uma tarefa trivial. Segundo \citeonline{Rose:2019}, a utilização destas ferramentas tecnológicas devem mudar a forma como os agricultores interagem com suas plantações, com isso, cada vez menos os trabalhadores precisarão atuar em atividades manuais, mudando a forma de como sempre interagiram com seu ofício. E isso pode dificultar a adesão a novas tecnologias. A utilização das técnicas relacionadas a agricultura de precisão demonstraram certa resistência por parte dos agricultores \cite{Rose:2019}. O uso em larga escala de Inteligência Artificial (IA), robótica e outras inovações emergentes, tem o claro potencial de causar consequências sociais não intencionais, imprevistas e indesejadas. Por esse motivo, é importante focar em uma parte fundamental para o sucesso da aplicação de novas tecnologias: o usuário.

Grande parte das tecnologias adotadas pela agricultura 4.0 está inserida (pelo menos em algum momento) em um sistema computacional. Sabe-se que um sistema computacional é basicamente composto pela \textit{entrada de dados}, o seu \textit{processamento} e a disponibilização dos seus \textit{resultados}. As duas últimas etapas dependem da qualidade da fonte de dados, ou seja, da entrada das informações \cite{Torres:2013}.

No âmbito agrícola, por mais automatizados que possam estar os processos de obtenções dos dados, com sensores automáticos ou análise de imagens, por exemplo, há dois principais pontos que podem aumentar ainda mais a resistência à adoção das tecnologias: \textit{indisponibilidade de equipamentos} e \textit{conhecimento empírico}. No cenário atual, não são todos os agricultores que podem destinar parte do seu faturamento para a aquisição de equipamentos modernos que facilitariam a coleta de dados. Além disso, existe uma variável de grande importância que deve ser inserida na equação da agricultura 4.0, a experiência do agricultor com a sua propriedade e a sua área de atuação. Segundo \citeonline{Rose:2019} aproximar o usuário final do processo do desenvolvimento de sistemas para a agricultura 4.0, pode ser uma estratégia eficiente para a adoção de novas tecnologias.

A motivação para o presente trabalho consiste no crescimento da agricultura 4.0, formada por diversos sistemas computacionais, bem como múltiplos usuários, que podem colaborar entre si de forma eficiente usando essas tecnologias para realizar tarefas. Dessa forma, métodos, técnicas e ferramentas da engenharia de \textit{software} devem ser estudados para verificar suas aplicações no projeto, implementação e avaliação de sistemas desse tipo.

\section{Revisão da literatura}
\label{sec:revisao_literatura}

O apoio a tomada de descisão.

A colaboração é uma questão importante a ser abordada, pois permite que múltiplos usuários, em diferentes plataformas de \textit{hardware} e \textit{software} e com diferentes tipos de representação da informação, trabalhem em um mesmo problema para resolução. Esse assunto é argumento na Seção \ref{sec:analise_critica_relevancia}

O usuário.

\section{Trabalhos Relacionados}
\label{sec:trabalhos_relacionados}

O trabalho de \citeonline{Zhai:2020} apresenta como desafios para os sistemas de apoio a decisão agrícola: a simplificação de interfaces gráficas dos usuários e utilização de conhecimento dos especialistas do domínio do sistema. Além disso, são abordadas alternativas às entradas de dados comuns, como por exemplo: a utilização do reconhecimento de voz ou a utilização gestos. Outro ponto importante está relacionado ao processo de extração das informações da fonte de dados. Modelos tridimensionais, análise de imagens ou vídeos, e imagens interativas podem ser pontos ainda a serem explorados no contexto em que este trabalho se aplica. Para completar, a colaboração entre usuários também é pouco explorada.

\citeonline{Gutierrez:2019} demonstram por meio da comparação de trabalhos, que a utilização de diversas fontes de informações de entradas é um ponto desejável para a automatização e autonomia na agricultura 4.0. As principais comparações estão relacionadas às formas de visualização de dados processados nesse contexto. As entradas dos dados são tratadas de forma superficial, bem como a colaboração entre os usuários. Os autores reafirmam a preocupação do presente trabalho em inserir diversas dimensões em uma mesma fonte de dados, e que o processo de obtenção de dados refinados pode aprimorar os resultados obtidos.

Nos trabalhos de \citeonline{Walling:2020} e \citeonline{Lundstrom:2018} é mostrada a importância da participação do usuário no processo de obtenção de dados em sistemas relacionados a agricultra 4.0. Profissionais com a experiência de seu domínio, são capazes de enriquecer toda a cadeia com a inferência de seu conhecimento nos sistemas de apoio a decisão. Com esses estudos, fica claro que a participação multidisciplinar em toda a cadeia dos sistemas de apoio a decisão podem fazer uma diferença expressiva nos resultados.

O sistema AgroDSS, explorado por \citeonline{Rupnik:2019}, é um exemplo de sistema em que a maioria dos dados coletados é baseada apenas em informações inseridas por meio de formulários de atributos textuais.

No trabalho de \citeonline{Fallon:2018} argumenta-se a importância de previsões climáticas sazonais para apoiar as decisões agrícolas de curto prazo, e planos de adaptação climática de longo prazo. Foi desenvolvida uma ferramenta de gerenciamento de terras (LMTool), um protótipo de serviço climático sazonal para gestores fundiários. Esta ferramenta fornece previsões de (1 a 3 meses à frente) para agricultores no sudoeste do Reino Unido, além previsões metereológicas específicas de 14 dias durante os meses de inverno, quando a habilidade das previsões sazonais é maior. Descreve-se ainda, os processos de desenvolvimento em conjunto com os agricultores.

No trabalho de \citeonline{Devitt:2017} é descrito um aplicatvo móvel para o apoio a tomada de decisões com relação ao manejo de plantas daninhas. A ferramenta interativa auxilia os produtores de algodão na exploração do impacto causado pela praga, além de traçar sugestões de estratégias individuais. A pesquisa aborda o desafio de envolver as partes interessadas na tomada de decisão de três maneiras: 1) reconhecendo as prioridades cognitivas individuais; 2) vizualiando o manejo científico de plantas daninhas com uma interface móvel atraente e 3) representando incertezas na decisão e o risco ponderado em relação às prioridades cognitivas.

O trabalho de \citeonline{Luvisi:2011} utiliza um método para combinar dados armazenados em um microchip RFID implantado dentro de plantas de videira e um GPS. O \textit{software} GIS foi usado para registrar coordenadas geográficas dos pontos detectados e para popular um banco de dados específico no qual as informações úteis para a fase de posicionamento sejam armazenadas. O produto final é um mapa digital acessível por dispositivos móveis ou uma aplicação desktop que representa o "vinhedo virtual". Nesta representação digital, cada planta de videira marcada por RFID pode ser selecionada, visualizada e editada.

Já no trabalho de \citeonline{Mudisshu:2016} argumenta-se sobre a diminuição da produção agrícola ano a ano no Japão, causada pela liberação das importações de produtos agrícolas. O projeto tem o objetivo de reduzir os custos nas aquisoções de dados meterológicos, com o desenvolvimento de um aplicativo para \textit{smartphone} para a visualização de dados.

No trabalho de \citeonline{Tan:2012} argumenta-se que os operadores de campo, podem coletar dados volumosos que fornecem em tempo real dados que representam as condições do campo em determinado momento. Destaca-se ainda, que é um grande desafio analisar todos estes dados de forma eficiênte e utilizá-los para melhorar as decisões agrícolas. A proposta é um \textit{software} para análise e visualização de dados com precisão, que conta com três características distintas: 1) um meta-modelo de componente de importação, capaz de importar dados em vários formatos a partir de uma variedade de dispositivos em diferentes configurações; 2) um processamento de dados orientado por fluxos de trabalho, no qual um usuário pode definir seus fluxos de trabalho e adicionar processamento de dados definido de forma personalizada operadores para uma aplicação específica; 3) uma arquitetura seguindo um modelo cliente-servidor que com suporte a uma variedade de dispositivos clientes, incluindo dispositivos móveis.

Por fim, no trabalho de \citeonline{Zheng:2017} destaca-se a demanda por aplicativos móveis como ferramentas de apresentação de dados. O trabalho busca projetar e desenvolver estratégias para uma apresentação mais amigável dos dados para seus usuários. 

Nos trabalhos, a exploração de métodos, técnicas e ferramentas da engenharia de software na agricultura 4.0 é superficial. No entanto, a participação dos usuários no desenvolvimento e a implementação de interfaces humano-computador para entrada e saída de informações são pontos mencionados.

\section{Justificativa}
\label{sec:justificativa}

No cenário agrícola, são vários os papéis dos profissionais que compõem toda a cadeia de produção. A obtenção dos dados utilizados nos sistemas computacionais de apoio à agricultura deve levar em consideração a diversidade de usuários, como agricultores, técnicos agrícolas, engenheiros agrônomos, entre outros. As experiências desses múltiplos profissionais poderiam ser utilizadas para os sistemas computacionais. Por exemplo, para uma nova praga identificada em uma folha de soja, o \textbf{agricultor} poderia inserir as informações relacionadas as suas experiências, como por exemplo se já houve alguma praga com indícios semelhantes aos encontrados, a quanto tempo, por quanto tempo, e como foi resolvido. Enquanto que o \textbf{engenheiro agrônomo} poderia inserir para esta mesma entrada informações extraídas de sua experiência em outras fazendas, \textbf{colaborando} para solucionar o problema de identificação dessa praga. Dessa forma consegue-se extrair diversas \textbf{dimensões} a partir de uma única entrada de dados \cite{Walling:2020}.

Segundo uma pesquisa divulgada pela Fundação Getúlio Vargas (FGV-SP), o Brasil tem hoje dois dispositivos digitais por habitante, incluindo \textit{smartphones}, computadores, \textit{notebooks} e \textit{tablets} \cite{FGV:2020}. Nesse sentido, o país teria aproximadamente 420 milhões de aparelhos digitais ativos. Dessa forma, fica bastante claro que estes dispositivos já fazem parte do dia a dia de muitas pessoas, inclusive dos que fazem parte de toda a cadeia agrícola, sejam agricultores de pequeno e médio portes.

A utilização de \textit{smartphones} e computadores, por serem equipamentos de baixo custo, propõe uma solução acessível, para um dos pontos mencionados em \citeonline{Rose:2019}, que argumentam que é necessário ampliar as noções de ``inclusão'' de forma responsável, para atender às diversas maneiras de como os agricultores devem interagir com as fazendas inteligentes. Estes dispositivos já fazem parte da rotina de trabalho do público alvo, não sendo necessários outros investimentos ou aquisições, e ainda assim, possibilitando a inclusão desse público em novas ferramentas criadas para a agricultura 4.0.

No cenário proposto pela agricultura 4.0, grande parte dos trabalhos evidenciam que as informações que alimentam os sistemas, podem ser obtidas de forma automatizada, por meio de sensores ou pela análise de imagens, por exemplo. Entretanto, nem sempre essa é uma realidade aplicável a todos, pelo alto custo e difícil aderência. Outros sistemas, mesmo que completos em termos de sensores e outros recursos tecnológicos, utilizam ainda, entrada manual de informações por parte dos seus usuários, normalmente por meio de formulários com campos pré definidos (comumente dados no formato texto). Sensores e outros recursos também exigem monitoramento de energia, manutenção e desempenho.

Além disso, presente projeto visa contribuir com determinados \textit{objetivos de desenvolvimento sustentável}, descritos pela Organização das Nações Unidas \cite{ONU:2020}. No documento da organização, são descritos 17 objetivos que buscam concretizar os direitos humanos e promover a igualdade. Para isso, tais objetivos e metas foram definidos com o intuito de estimular ações para os próximos 15 anos, em áreas de importância crucial para a humanidade e para o planeta. Dentre os objetivos, dois deles têm relação direta com este trabalho:

\begin{itemize}
	\item \textit{Objetivo 2: fome zero e agricultura sustentável} - Aumentar a eficiência da produção agrícola é consequência direta da proposta deste trabalho.
	\item \textit{Objetivo 12: consumo e produções sustentáveis} - Neste quesito, o trabalho contribui com a proposta de uma solução eficiente, acessível (utilizando um \textit{smartphone} como equipamento, por exemplo) e sem a necessidade da aquisição de novos equipamentos para integrações tecnológicas.
\end{itemize}

\section{Objetivos}
\label{sec:objetivos}

\subsection{Objetivo geral}
\label{subsec:objetivo_geral}

O objetivo deste projeto é partir da exploração de métodos, técnicas ou ferramentas de engenharia de software, no contexto da agricultura 4.0, apoiar o desenvolvimento de sistemas. Para isso, será conduzido um estudo em engenharia de software para projetar, implementar e avaliar sistemas computacionais colaborativos, para auxiliar os usuários na realização de tarefas.

\subsection{Objetivos específicos}
\label{subsec:objetivos_especificos}

\begin{itemize}
	\item Pesquisar e avaliar sistemas de apoio à tomada de decisão e colaborativos, em especial os direcionados para agricultura 4.0, analisando seus pontos positivos e negativos;
	\item Investigar e entender o perfil dos usuários inseridos no contexto agrícola;
	\item Criar uma nova abordagem que direcione o desenvolvimento de sistemas computacionais, com foco na agricultura 4.0;
	\item Avaliar a nova abordagem através da realização da implementação da proposta, seguida de experimentos e comparações com outras abordagem existentes;
\end{itemize}

\section{Problema a ser Investigado}
\label{sec:problema_investigado}

O principal problema de pesquisa a ser tratado por este trabalho está relacionado a questões sobre como projetar, implementar e avaliar sistemas computacionais colaborativos eficientes para apoiar a tomada de decisão dos usuários no domínio da agricultura 4.0, usando métodos, técnicas e ferramentas da Engenharia de Software.

Os seguintes tópicos resumem em linhas gerais, quais são os problemas a serem explorados durante o desenvolvimento do projeto:

\subsection{Acesso às Tecnologias}
\label{subsec:acesso_tecnologias}

Grande parte dos trabalhos na literatura, como demonstrado em \cite{Massruha:2017}, possuem o foco nos processos de análise e apoio a decisão. Estes trabalhos de pesquisa, normalmente, são suportados por grandes instituições (como universidades e empresas de grande porte) que provêm recursos (como sensores ou drones) para a captação de dados no desenvolvimento dos experimentos. Com tais recursos, os experimentos tendem a focar nos processos que geram os resultados e consequentemente demonstram a eficiência e a eficácia das técnicas propostas pela agricultura 4.0. Entretanto essa não é a realidade de grande parte dos agricultores, como argumenta \cite{Rose:2019}. Os estudos propostos por este trabalho, ajudam a resolver o problema da falta de acessibilidade aos recursos necessários para obtenção de dados, permitindo, mesmo que de uma forma menos eficiente e automatizada, a utilização de tecnologias da agricultura 4.0 por qualquer usuário da cadeia agrícola que possua apenas um \textit{smartphone}.

\subsection{Participação Ativa na Colaboração}
\label{subsec:participacao_ativa_colaboracao}

No âmbito agrícola, é bastante comum grandes áreas de plantio. Mapear completamente uma área com dispositivos automatizados poderia gerar um custo muito grande para os agricultores, principalmente com a necessidade de manutenção dos equipamentos que captam as informações dos campos, bem como o consumo de energia elétrica por estes dispositivos. Nesse cenário o presente trabalho também atua na pesquisa de alternativas que possibilitem envolver os fazendeiros nos processos de obtenção dos dados e análise por meio de colaboração. Isso poderia ser feito utilizando a sua experiência e conhecimento de sua propriedade para intuitivamente direcionar a coleta e análise de dados em áreas que pudessem apresentar uma anomalia.

\subsection{Arguição dos Especialistas}
\label{subsec:arquicao_especialistas}

A utilização de recursos automatizados, como drones ou sensores, tendem a facilitar muito o processo de obtenção de dados em uma fazenda que aplica técnicas da agricultura 4.0. Tratando de cenários nos quais essa já seja uma realidade, estes recursos automatizados normalmente apenas armazenam uma representação da realidade obtida em algum instante. Por exemplo, um sensor de umidade, poderia registrar os índices de umidade do solo a cada hora. Ou ainda, um drone, poderia fotografar uma área com baixa produtividade. Estes dados não levam em consideração alguns aspectos que poderiam ser enriquecidos com a arguição de um especialista, por exemplo um agrônomo. Por este motivo, este trabalho propõe a investigação de técnicas que possibilitem a inclusão de novas dimensões na coleta de dados, unindo por exemplo, a obtenção automática da umidade do solo com informações e experiências de agrônomos especialistas, que poderiam, por exemplo, informar que apesar de uma leitura de umidade ter ficado fora da expectativa, isso foi advento de uma anormalidade, por ele observada. 

Como há mais de um especialista em uma mesma tarefa, regras de colaboração devem ser especificadas. Adicionalmente, métodos, técnicas e ferramentas de Engenharia de Software para o desenvolvimento do sistema computacional que permitirá a colaboração deverão ser estudadas, auxiliando no projeto, implementação e avaliação.

Para os cenários nos quais se utiliza a captação automática dos dados, boa parte das abordagens visto em \cite{Massruha:2017} fazem essa captura de forma contínua e interruptamente, propondo abordagens que tratem grandes volumes de dados. Esta abordagem pode gerar grandes custos, inserindo, por exemplo, dados muitos parecidos que não ajudam na inferência nos resultados dos sistemas de apoio a decisão. Ao utilizar tratativas com mais dimensões, criando por exemplo visões diferentes de agricultores, técnicos agrícolas ou engenheiros agrônomos sobre uma mesma entrada de dados, poderia ajudar na obtenção de dados mais refinados e consequentemente na economia de recursos para manter a coleta automatizada.

\subsection{Interações Manuais}
\label{subsec:interacoes_manuais}

Quando se trata de agricultura, há diversos tipos de cenários. Existem diversos tipos de plantações: soja, milho, café, entre outros. Cada plantação possui suas particularidades, em termos de clima e solo. Em cenários nos quais uma plantação apresente vegetações mais densas, entende-se como um problema a captação automática de evidências que comprovem a presença de algum tipo de praga. Essencialmente, a presença dos agricultores e fazendeiros, no dia a dia, analisando as plantações é essencial para o acompanhamento do seu processo produtivo. Uma vegetação densa, poderia esconder uma praga de uma foto aérea feita por um drone, por exemplo. No trabalho proposto, busca-se uma alternativa a este cenário, permitindo a inclusão e confirmação manual dessas informações, que podem ser importantes para análise de inferências de apoio a decisão.

\subsection{Experiência e Interface para Colaboração}
\label{subsec:experiencia_interface_colaboracao}

A experiência do usuário com um dispositivo computacional pode influenciar diretamente na adoção de novas tecnologias ou sistemas. É bastante comum a resistência por parte de novos usuários na adoção de sistemas que ainda não tiverem contato. Como objeto de estudo deste trabalho, propõe-se ainda, descobertas relacionadas ao entendimento do perfil do público alvo, representando os usuários e suas peculiaridades. Criando, dessa forma, interfaces que possam ser objetivas, claras, atrativas e principalmente funcionais. Outro desafio a ser explorado é a interação multi-dimensional da entrada e análise de dados, que propõe-se ser refinada, simultaneamente, por diversos colaboradores e que resultem em uma única fonte de dados, possibilitando a inferência de apoio a decisão extraídas de dados refinados e alinhados com as realidades do cenário agrícola apontado por parte da cadeia que já atua na manutenção desta propriedade.

\subsection{Características de sistemas colaborativos e seus possíveis problemas}
\label{subsec:caracteristicas_problemas}

A seguir lista-se características de sistemas colaborativos (gerais e específicos), relacionadas entre si, e seus possíveis problemas no contexto da agricultura 4.0:

\begin{itemize}
	\item \textbf{atualização em tempo real} - Problemas: múltiplos acessos simultâneos; necessidade de exibição de alertas e erros sobre atrasos; regras de funcionamento (entrada, processamento e saída);
	\item \textbf{múltiplos dispositivos para entrada e saída de dados}. Por exemplo: \textit{smartphones}, \textit{desktops}, sensores e robôs. Problemas: integração; interoperabilidade; escalabilidade de dispositivos; tolerância a falhas - possibilidade de substituição de sensores danificados;
	\item \textbf{múltiplos usuários}. Por exemplo: agricultores, técnicos agrícolas, engenheiros agrônomos. Problemas: usabilidade (facilidade de uso, segurança, satisfação, acessibilidade, memorização do usuário), envolvendo interfaces humano-computador e formas de interação; definição de grupos de usuários; carga cognitiva (muitas tecnologias e formas de uso); escalabilidade de usuários; regras de uso; atribuição de significado e interpretação das informações;
	\item \textbf{múltiplas plataformas das fontes de entrada}. Por exemplo: diversas plataformas de \textit{smartphones}. Problemas: integração; interoperabilidade; usabilidade (facilidade de uso, segurança, satisfação, acessibilidade, memorização do usuário), envolvendo interfaces humano-computador e formas de interação;
	\item \textbf{múltiplos tipos de informações de entrada}. Por exemplo: imagens, voz, texto e vídeos. Problemas: armazenamento e processamento (imagens podem ocupar mais espaço), usabilidade (facilidade de uso, segurança, satisfação, acessibilidade, memorização), envolvendo interfaces humano-computador e formas de interação; carga cognitiva (muitas tecnologias e formas de uso;
	\item \textbf{múltiplos tipos de informações de saída}. Por exemplo: imagens, áudio, texto, modelos tridimensionais e vídeos. Problemas: armazenamento e processamento, usabilidade (facilidade de uso, segurança, satisfação, acessibilidade, memorização), envolvendo interfaces humano-computador e formas de interação; carga cognitiva (muitas tecnologias e formas de uso);
	\item \textbf{múltiplas informações sobre o domínio}. Por exemplo: planta, clima, pragas, ervas daninhas, doenças. Problemas: especificação de subdomínios; granularidade (sistema incorpora informações em atualizações); relação dos subdomínios – planta e clima, planta, pragas e clima);
	\item \textbf{informações de entradas de múltiplos usuários}. Por exemplo: agricultores, técnicos agrícolas, engenheiros agrônomos. Problemas: problemas de conexão com a Internet em regiões afastadas; de consumo de energia elétrica; de prioridade dependendo do nível do usuário e do nível de grupo de usuários do mesmo nível; de necessidade de diferentes interfaces humano-computador e formas de interação; de necessidade de definição de níveis e grupos; de processamento e armazenamento local, envolvendo o que pode ser processado e armazenado localmente de acordo com os recursos de hardware e software, o consumo de energia elétrica, de maneira a não prejudicar a entrada de novas informações
	\item \textbf{informações de saídas para múltiplos usuários}. Por exemplo: agricultores, técnicos agrícolas, engenheiros agrônomos. Problemas: de usabilidade (facilidade de uso, segurança, satisfação, acessibilidade, memorização), preferências dos usuários, envolvendo interfaces humano-computador e formas de interação);
	\item \textbf{informações de entradas de múltiplos sensores e robôs}. Problemas: conexão com a Internet em regiões afastadas; de nível de autonomia; de consumo de energia elétrica; de processamento e armazenamento local, envolvendo o que pode ser processado e armazenado localmente de acordo com os recursos de hardware e software, o consumo de energia elétrica, de maneira a não prejudicar a entrada de novas informações)
	\item \textbf{priorização das informações do domínio}. Por exemplo: o processamento de determinadas informações tem prioridade. Por exemplo, uma atividade de colheita está em andamento e o processamento de informações sobre o clima tem prioridade sobre o processamento de informações sobre pragas, no entanto, pode-se perder inferências. Problemas: necessidade da participação de usuários para definição de prioridades; da inviabilidade devido a quantidade de informações e necessidade de agrupamento (\textit{clusters}) de informações por data, região, clima, tipo da informação, fonte, ou combinação (região e clima); flexibilidade na combinação de informações)
	\item \textbf{sincronização das informações de entrada na nuvem (\textit{Cloud Computing}), que  envolve:}
		\begin{itemize}
			\item \textbf{priorização da informação a ser processada: a mesma informação é recebida pelo servidor ao mesmo tempo de diferentes fontes}. Por exemplo: do sensor A e do usuário 1. Problemas: identificação das similaridades; definição de prioridades com a participação de usuários;
			\item \textbf{complementação da informação: a mesma informação é recebida pelo servidor ao mesmo tempo de diferentes fontes}. Exemplo: sensor A e do usuário 1), mas de diferentes tipos (texto, imagem e voz), e elementos podem ser extraídos para completar ou confirmar a informação. Problemas: identificação das diferenças, extração e complemento; custo/benefício da  confirmação em termos de processamento, recursos utilizados;
			\item \textbf{completude da informação}. Por exemplo: a informação é recebida pelo servidor, mas incompleta, como por exemplo, uma imagem incompleta. Problemas: identificação das lacunas e correção, por aproximação, interpolação, utilização de um outro tipo de informação, adoção de um padrão, emissão de avisos para usuários sobre a necessidade de completar a informação – dependência de conexão com a Internet, energia elétrica, processamento; custo/benefício de cada forma de completude);
			\item \textbf{informação adicional automática}. Por exemplo: uma informação adicional é adicionada automaticamente em uma informação capturada pelo usuário. Por exemplo, o usuário captura uma imagem usando o \textit{smartphone}, e o sistema busca automaticamente a geolocalização e o clima. Problemas: conexão com a Internet em regiões afastadas; de consumo de energia elétrica; de processamento e armazenamento local, envolvendo o que pode ser processado e armazenado localmente de acordo com os recursos de hardware e software, o consumo de energia elétrica, de maneira a não prejudicar a entrada de novas informações);
			\item \textbf{classificação da informação após o processamento}. Por exemplo: informação foi útil, parcialmente útil ou inútil, atribuindo nível de qualidade para a informação. Problemas: custo/benefício para a classificação; de necessidade da participação de usuários; da inviabilidade devido a quantidade de informações e necessidade de agrupamento (\textit{clusters}) de informações por data, região, clima, tipo da informação, fonte, ou combinação (região e clima); flexibilidade na combinação de informações).
		\end{itemize}
\end{itemize}

\subsection{Cenário Possível}
\label{subsec:cenario_possivel}

Para ilustrar as possibilidades a serem exploradas, descreve-se a utilização de um aplicativo em um cenário hipotético, no qual:

\begin{enumerate}
	\item Um fazendeiro, motivado pelo desconhecimento de uma praga, captura uma imagem de um inseto em uma planta de soja usando seu \textit{smartphone};
	\item A imagem pode receber informações adicionais no próprio \textit{smartphone}, tais como: anotações da quantidade, nome do inseto, enfatizando o tamanho do inseto e o estádio fenológico da planta;
	\item Ainda na imagem, pode ser possível a inclusão de anotações ou destaques em pontos que pudessem ser circulados, pelo fazendeiro, demonstrando pontos de atenção que pudessem ser analisados;
	\item Um engenheiro agrônomo, parceiro deste fazendeiro, \textbf{colabora} analisando a mesma imagem em seu próprio perfil do aplicativo, de forma remota, inserindo e/ou complementando informações sobre o cenário investigado;
	\item Todas as imagens poderiam servir de fonte de dados: a original, a com as anotações do fazendeiro e do engenheiro agrônomo;
	\item Por conta de uma possível dificuldade de conexão (muito comum no cenário rural) o envio das informações poderia não acontecer imediatamente. Mas a operação poderia ser marcada como sucesso, deixando essa atividade uma fila de execução em \textit{background};
	\item O próprio sistema registraria as coordenadas de geolocalização, bem como o momento (horário/dia) dessa captura ou oferecer a possibilidade de inserção manual dessas informações (prevendo a instabilidade da conexão);
	\item Ao receber as informações, um sistema computacional em nuvem, pode relacionar as imagens recebidas com uma ou mais imagens aéreas, capturadas por veículos aéreos não tripulados da mesma região onde está a planta, ou regiões adjacentes em um ou mais momentos anteriores, buscando por insetos. O sistema computacional poderia registrar dia e horário de captura da imagem, de modificação, de envio com sucesso, de colaboração, de usuário que colaborou;
	\item O sistema computacional poderia ainda utilizar utilizar bases de informações compartilhadas entre diferentes regiões e usuários aumentando o refinamento na inferência das informações de apoio a decisão.
\end{enumerate}

\section{Metodologia}
\label{sec:metodologia}

As atividades para a realização da pesquisa incluem:

\begin{enumerate}
	\item Revisão da literatura e análise crítica, buscando explorar os pontos positivos e negativos dos trabalhos estudados, e quais seriam as correlações com a nova proposta a ser definida;
	\item Criação e execução de um plano de análise das personas inseridas no contexto da agricultura 4.0. Nesta atividade, destaca-se a importância do trabalho em conjunto com a Embrapa \footnote{Empresa Brasileira de Pesquisa Agropecuária, situada em Londrina, trabalha principalmente com a produção da soja. A Embrapa é uma empresa pública de pesquisa vinculada ao ministério da agricultura, pecuária e bbastecimento do Brasil. O seu principal objetivo é a viabilização de soluções provenientes de pesquisas, além do desenvolvimento para a sustentabilidade da agricultura em benefício da sociedade brasileira.};
	\item Criação de uma proposta inicial conceitual através do desenvolvimento e evolução de protótipos de sistemas computacionais colaborativos com foco na agricultura 4.0 e suas análises;
	\item Aplicação da proposta conceitual através da criação de um sistema computacional colaborativo e de sua publicação para o público final;
	\item Planejamento, organização e execução de experimentos envolvendo usuários. Com a utilização de questionários para coletar opiniões referentes as percepções sobre a utilização do sistema. Outras abordagens para obtenção de resultados podem ser empregadas:
	\begin{itemize}
		\item Testes A/B;
		\item Análise de Abandono;
		\item Avaliação do Aplicativo na Loja;
		\item Análise de Usuários Ativos do Aplicativo.
	\end{itemize}
	\item Análise de resultados utilizando testes estatísticos para verificar a significância das diferenças entre os grupos amostrais comparados (Friedman, ANOVA – Análise da Variância, t-test, Wilcoxon, Mann-Whitnney etc); e estatística descritiva (principalmente gráficos). Os testes estatísticos dependem das características dos dados (normalidade da distribuição, homogeneidade das variâncias, independência e aleatoriedade na coleta etc) e oferecem intervalos de confiança para a análise.
\end{enumerate}

\section{Atividades}
\label{sec:atividades}

O plano de trabalho é composto pelas seguintes atividades principais, dispostas no cronograma apresentado na Tabela \ref{tab:cronograma}, separadas por quadrimestres de cada ano:

\begin{enumerate}
	\item Obtenção de créditos, cursando disciplinas;
	\item Estudo de métodos para mapeamento ou revisão sistemática, permitindo o levantamento de trabalhos na literatura, bem como métodos, técnicas e ferramentas da Engenharia de Software para o desenvolvimento de protótipos;
	\item Estudo do estado da arte, considerando métodos, técnicas e ferramentas de Engenharia de Software para projetar, implementar e avaliar sistemas computacionais colaborativos com dispositivos móveis voltados para a agricultura 4.0;
	\item Especificação de cenários para utilização de métodos, técnicas e ferramentas de Engenharia de Software no projeto, implementação e avaliação de colaboração no âmbito da agricultura 4.0. A adaptação de métodos, técnicas e ferramentas pode ser necessária;
	\item Exame de proficiência de língua estrangeira;
	\item Exame de qualificação;
	\item Especificação de um sistema computacional colaborativo ou módulos desse sistema, indicando a arquitetura e principais desafios no desenvolvimento. Pode ser necessária a aplicação de métodos, técnicas e ferramentas selecionados, ou selecionados e adaptados, no processo de desenvolvimento;
	\item Implementação de protótipos para validação, com a aplicação de métodos, técnicas e ferramentas selecionados ou selecionados e adaptados;
	\item Redação de artigos científicos a serem submetidos aos principais eventos e periódicos da área de interesse;
	\item Avaliação da aplicação de métodos, técnicas e ferramentas. Testes envolvendo usuários podem ser necessários, especialmente para a avaliação do sistema computacional colaborativo ou módulo desse sistema;
	\item Redação da tese;
	\item Defesa.
\end{enumerate}

\section{Resultados Esperados}
\label{sec:resultados_esperados}

Pretende-se com o estudo levantar métodos, técnicas e ferramentas de Engenharia de Software para o projeto, implementação e avaliação de sistemas computacionais colaborativos eficientes no contexto da agricultura 4.0, permitindo que usuários realizem tarefas para aumentar a produção agrícola de forma sustentável.

Espera-se, por fim, a submissão de artigos para conferências e periódicos científicos da área de Ciência da Computação para publicação, como o \textit{ACM Computing Surveys}, \textit{Computer Graphics and Applications}, \textit{Journal on Interactive Systems}, \textit{Computers and Electronics in Agriculture}, \textit{Symposium on Applied Computing}, Simpósio Brasileiro de Engenharia de Software.

\section{Cronograma}
\label{sec:cronograma}

As atividades e os períodos serão ajustados conforme o cronograma do Programa de Pós-Graduação em Informática da UFPR.

\begin{table}[htbp]
	\centering
		\begin{tabular}{|c|c|c|c|c|c|c|c|c|c|c|c|c|}
		\hline
		\multirow{2}{*}{AT} & \multicolumn{3}{c|}{\textbf{Ano 1}} & \multicolumn{3}{c|}{\textbf{Ano 2}} & \multicolumn{3}{c|}{\textbf{Ano 3}} & \multicolumn{3}{c|}{\textbf{Ano 4}} \\ \cline{2-13} 
												& Q1         & Q2         & Q3        & Q1         & Q2         & Q3        & Q1         & Q2         & Q3        & Q1         & Q2         & Q3        \\ \hline
		1                   & •          & •          & •         &            &            &           &            &            &           &            &            &           \\ \hline
		2                   &            &            &           & •          &            &           &            &            &           &            &            &           \\ \hline
		3                   &            &            &           & •          & •          & •         & •          &            &           &            &            &           \\ \hline
		4                   &            &            &           &            & •          &           &            &            &           &            &            &           \\ \hline
		5                   &            &            &           &            & •          &           &            &            &           &            &            &           \\ \hline
		6                   &            &            &           &            & •          &           &            &            &           &            &            &           \\ \hline
		7                   &            &            &           &            & •          &           &            &            &           &            &            &           \\ \hline
		8                   &            &            &           &            &            & •         & •          & •          &           &            &            &           \\ \hline
	  9                   &            &            &           &            &            & •         & •          & •          & •         & •          & •          & •         \\ \hline
		10                  &            &            &           &            &            &           &            &            &           & •          &            &           \\ \hline
		11                  &            &            &           &            &            &           &            &            &        &           & •          & •         \\ \hline
		12                  &            &            &           &            &            &           &            &            &           &            &            & •         \\ \hline
		\end{tabular}
	\caption{Cronograma de Execução das Atividades}
	\label{tab:cronograma}
\end{table}

\bibliography{project}

\end{document}